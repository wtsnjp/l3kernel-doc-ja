% +++
% sequence = ["latex", "dvipdf"]
% latex = "uplatex"
% clean_files = [
%   "%B.aux", "%B.dvi", "%B.glo", "%B.hd", "%B.idx", "%B.ind",
%   "%B.ilg", "%B.log", "%B.out", "%B.synctex.gz",
% ]
% +++
\documentclass[uplatex,dvipdfmx,full,kernel]{wtpl3doc}
\usepackage{interface3-ja}

\begin{document}

%\title{The \pkg{l3names} package\\ Namespace for primitives}
\title{\pkg{l3names}パッケージ\\ プリミティブのための名前空間}
%\author{%
% The \LaTeX3 Project\thanks
%   {%
%     E-mail:
%       \href{mailto:latex-team@latex-project.org}
%         {latex-team@latex-project.org}%
%   }%
%}
\author{%
 \LaTeX3プロジェクト\thanks
   {%
     E-mail:
       \href{mailto:latex-team@latex-project.org}
         {latex-team@latex-project.org}%
   }%
}
%\date{Released 2020-01-22}
\date{バージョン 2020-01-22}

\maketitle

\begin{documentation}

%\section{Setting up the \LaTeX3 programming language}
\section{\LaTeX3プログラミング言語のセットアップ}

%This module is at the core of the \LaTeX3 programming language. It
%performs the following tasks:
このモジュールは\LaTeX3プログラミング言語のコアにあたり,以下のタスクを
実行します:
%
\begin{itemize}
%  \item defines new names for all \TeX{} primitives;
%  \item switches to the category code r{\'e}gime for programming;
%  \item provides support settings for building the code as a \TeX{} format.
  \item すべての\TeX プリミティブについて,新しい名前を定義
  \item カテゴリーコード環境をプログラミング用に変更
  \item コードを\TeX フォーマットとしてビルドするためのサポートを提供
\end{itemize}

%This module is entirely dedicated to primitives, which should not be
%used directly within \LaTeX3 code (outside of \enquote{kernel-level}
%code). As such, the primitives are not documented here: \emph{The
%\TeX{}book}, \emph{\TeX{} by Topic} and the manuals for \pdfTeX{},
%\XeTeX{}, \LuaTeX{}, \pTeX{} and \upTeX{} should be consulted for
%details of the primitives. These are named
%\cs[no-index]{tex_\meta{name}:D}, typically based on the primitive's
%\meta{name} in \pdfTeX{} and omitting a leading |pdf| when the
%primitive is not related to pdf output.
このモジュールは\TeX プリミティブを扱うためのものです.プリミティブは
(「カーネルレベル」のコードを除き)\LaTeX3コードの中では直接使用する
べきではありません.そのため\TeX のプリミティブについてはここでは解説
しません.そうした解説については『\TeX ブック』や『{\TeX} by Topic』,
あるいは\pdfTeX, \XeTeX, \LuaTeX, \pTeX, \upTeX のマニュアルを参照して
ください.\LaTeX3において各\TeX プリミティブは,典型的にはその名前
\meta{name}に基づいて\cs[no-index]{tex_\meta{name}:D}という形で与えられ
ます.ただし\pdfTeX のプリミティブについては,プレフィックスの|pdf|は
機能的にPDF出力と関係がない限り省略されます.

\end{documentation}

\end{document}
