% +++
% sequence = ["latex", "dvipdf"]
% latex = "uplatex"
% clean_files = [
%   "%B.aux", "%B.dvi", "%B.glo", "%B.hd", "%B.idx", "%B.ind",
%   "%B.ilg", "%B.log", "%B.out", "%B.synctex.gz",
% ]
% +++
\documentclass[uplatex,dvipdfmx,full,kernel]{wtpl3doc}
\usepackage{interface3-ja}

\begin{document}

%\title{The \pkg{l3bootstrap} package\\ Bootstrap code}
\title{\pkg{l3bootstrap}パッケージ\\ ブートストラップコード}
%\author{%
% The \LaTeX3 Project\thanks
%   {%
%     E-mail:
%       \href{mailto:latex-team@latex-project.org}
%         {latex-team@latex-project.org}%
%   }%
%}
\author{%
 \LaTeX3プロジェクト\thanks
   {%
     E-mail:
       \href{mailto:latex-team@latex-project.org}
         {latex-team@latex-project.org}%
   }%
}
%\date{Released 2020-01-22}
\date{バージョン 2020-01-22}

\maketitle

\begin{documentation}

%\section{Using the \LaTeX3 modules}
\section{\LaTeX3モジュールを使用する}

%The modules documented in \file{source3} are designed to be used on top of
%\LaTeXe{} and are loaded all as one with the usual |\usepackage{expl3}| or
%|\RequirePackage{expl3}| instructions. These modules will also form the
%basis of the \LaTeX3 format, but work in this area is incomplete and not
%included in this documentation at present.
\file{source3}で説明されているモジュールは,\LaTeXe 上で
|\usepackage{expl3}|または|\RequirePackage{expl3}|によって一括読み込みされる
ことを意図して設計されています.これらのモジュールは\LaTeX3フォーマットの
基礎になる予定ですが,その部分については未完成であり,したがって現状この文書
にはその説明は含まれていません.

%As the modules use a coding syntax different from standard
%\LaTeXe{} it provides a few functions for setting it up.
\pkg{expl3}のモジュールは標準的な\LaTeXe のそれとは異なるシンタックスを用いる
ので,このモジュールではそうした環境をセットアップするための関数をいくつか
提供しています.

\begin{function}[updated = 2011-08-13]{\ExplSyntaxOn, \ExplSyntaxOff}
  \begin{syntax}
    \cs{ExplSyntaxOn} \meta{code} \cs{ExplSyntaxOff}
  \end{syntax}
%  The \cs{ExplSyntaxOn} function switches to a category code
%  r\'{e}gime in which spaces are ignored and in which the colon (|:|)
%  and underscore (|_|) are treated as \enquote{letters}, thus allowing
%  access to the names of code functions and variables. Within this
%  environment, |~| is used to input a space. The \cs{ExplSyntaxOff}
%  reverts to the document category code r\'{e}gime.
  \cs{ExplSyntaxOn}関数はカテゴリーコード環境の切り替えを行います.具体的には
  空白文字は無視されるようになり,またコロン (\code{:}) とアンダースコア
  (\code{_}) は\enquote{letters}として扱われるようにすることによって関数や
  変数へのアクセスを可能にします.この環境下では,空白文字の入力にはチルダ
  (|~|) を使用します.\cs{ExplSyntaxOff}関数はカテゴリーコード環境を元の状態に
  戻します.
\end{function}

\begin{function}[updated = 2017-03-19]
  {\ProvidesExplPackage, \ProvidesExplClass, \ProvidesExplFile}
  \begin{syntax}
    |\RequirePackage{expl3}| \\
    \cs{ProvidesExplPackage} \Arg{package} \Arg{date} \Arg{version} \Arg{description}
  \end{syntax}
%  These functions act broadly in the same way as the corresponding
%  \LaTeXe{} kernel
%  functions \tn{ProvidesPackage}, \tn{ProvidesClass} and
%  \tn{ProvidesFile}. However, they also implicitly switch
%  \cs{ExplSyntaxOn} for the remainder of the code with the file. At the
%  end of the file, \cs{ExplSyntaxOff} will be called to reverse this.
%  (This is the same concept as \LaTeXe{} provides in turning on
%  \tn{makeatletter} within package and class code.) The \meta{date} should
%  be given in the format \meta{year}/\meta{month}/\meta{day}. If the
%  \meta{version} is given then it will be prefixed with \texttt{v} in
%  the package identifier line.
  これらの関数は概ね\LaTeXe カーネルの\tn{ProvidesPackage},
  \tn{ProvidesClass}, \tn{ProvidesFile}と同じように動作します.しかし,
  これらはファイルの終端までを\cs{ExplSyntaxOn}状態へと暗黙的に切り替え
  ます.ファイルの終端では\cs{ExplSyntaxOff}が呼ばれて元の状態へと戻り
  ます(これは\LaTeXe がパッケージやクラスのコードでは\tn{makeatletter}%
  を有効にするのと同様です).なお\meta{date}は\meta{year}/\meta{month}%
  /\meta{day}のフォーマットで与えられる必要があります.\meta{version}が
  与えられた場合には,ログのパッケージ読み込みを示す箇所には\code{v}が
  前置されます.
\end{function}

\begin{function}[updated = 2012-06-04]{\GetIdInfo}
  \begin{syntax}
    |\RequirePackage{l3bootstrap}|
    \cs{GetIdInfo} |$Id:| \meta{SVN info field} |$| \Arg{description}
  \end{syntax}
%  Extracts all information from a SVN field. Spaces are not
%  ignored in these fields. The information pieces are stored in
%  separate control sequences with \cs{ExplFileName} for the part of the
%  file name leading up to the period, \cs{ExplFileDate} for date,
%  \cs{ExplFileVersion} for version and \cs{ExplFileDescription} for the
%  description.
  すべての情報をSVNフィールドから取得します.このフィールドでは空白文字は
  無視されません.各種の情報はそれぞれ別の制御綴に格納されます.具体的には
  ファイル名は\cs{ExplFileName},日付は\cs{ExplFileDate},バージョン番号は
  \cs{ExplFileVersion},説明は\cs{ExplFileDescription}に格納されます.
\end{function}

%To summarize: Every single package using this syntax should identify
%itself using one of the above methods. Special care is taken so that
%every package or class file loaded with \tn{RequirePackage} or similar
%are loaded with usual \LaTeXe{} category codes and the \LaTeX3 category code
%scheme is reloaded when needed afterwards. See implementation for
%details. If you use the \cs{GetIdInfo} command you can use the
%information when loading a package with
まとめると,\pkg{expl3}シンタックスを使用するすべてのパッケージは自身を
上記のいずれかの方法で宣言する必要があります.あらゆるパッケージまたは
クラスについて,\tn{RequirePackage}その他の方法で\LaTeXe のカテゴリー
コード環境下で読み込まれた際にも\LaTeX3カテゴリーコード環境に変更される
よう特別な処理が行なわれます.詳細については実装パートを確認してください.
\cs{GetIdInfo}コマンドを使用した場合には,以下のようにすることでその情報
を読み込みの際に活用することができます:
%
\begin{verbatim}
  \ProvidesExplPackage{\ExplFileName}
    {\ExplFileDate}{\ExplFileVersion}{\ExplFileDescription}
\end{verbatim}

\end{documentation}

\end{document}
