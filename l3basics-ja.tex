% +++
% sequence = ["latex", "dvipdf"]
% latex = "uplatex"
% clean_files = [
%   "%B.aux", "%B.dvi", "%B.glo", "%B.hd", "%B.idx", "%B.ind",
%   "%B.ilg", "%B.log", "%B.out", "%B.synctex.gz",
% ]
% +++
\documentclass[uplatex,dvipdfmx,full,kernel]{wtpl3doc}
\RequirePackage{interface3-ja}

\begin{document}

%\title{The \pkg{l3basics} package\\ Basic definitions}
\title{\pkg{l3basics}パッケージ\\ 基本的な定義}
%\author{%
% The \LaTeX3 Project\thanks
%   {%
%     E-mail:
%       \href{mailto:latex-team@latex-project.org}
%         {latex-team@latex-project.org}%
%   }%
%}
\author{%
 \LaTeX3プロジェクト\thanks
   {%
     E-mail:
       \href{mailto:latex-team@latex-project.org}
         {latex-team@latex-project.org}%
   }%
}
%\date{Released 2020-01-22}
\date{バージョン 2020-01-22}

\maketitle

\begin{documentation}

%As the name suggest this package holds some basic definitions which
%are needed by most or all other packages in this set.
名前が示すとおり,このモジュールは\pkg{expl3}に含まれるほとんどの
パッケージが必要とするいくつかの基本的な定義を行うものです.

%Here we describe those functions that are used all over the place. With
%that we mean functions dealing with the construction and testing of
%control sequences. Furthermore the basic parts of conditional
%processing are covered; conditional processing dealing with specific
%data types is described in the modules specific for the respective
%data types.
ここで\pkg{expl3}全体で用いられるそれらの基本的な関数を説明します.
具体的には,制御綴の作成やテストに関わる関数です.加えて,このモジュール
は基本的な条件分岐の機能提供も行います.ただし特定のデータ型に固有の
条件分岐については,ここではなくそれぞれのデータ型を扱うモジュールに
おいて説明されます.

%\section{No operation functions}
\section{ノーオペレーション関数}

\begin{function}[EXP]{\prg_do_nothing:}
  \begin{syntax}
    \cs{prg_do_nothing:}
  \end{syntax}
%  An expandable function which does nothing at all: leaves nothing
%  in the input stream after a single expansion.
  完全に何もしない,展開可能な関数です.1回展開すると空文字列になり,
  入力ストリームには何も残しません.
\end{function}

\begin{function}{\scan_stop:}
  \begin{syntax}
    \cs{scan_stop:}
  \end{syntax}
%  A non-expandable function which does nothing. Does not vanish on
%  expansion but produces no typeset output.
  何もしない展開不能な関数です.展開操作で消えることはありませんが,
  組版結果には何も作用しません.
\end{function}

%\section{Grouping material}
\section{グループ}

\begin{function}{\group_begin:, \group_end:}
  \begin{syntax}
    \cs{group_begin:}
    \cs{group_end:}
  \end{syntax}
%  These functions begin and end a group for definition purposes.
%  Assignments are local to groups unless carried out in a global
%  manner. (A small number of exceptions to this rule will be noted
%  as necessary elsewhere in this document.) Each \cs{group_begin:}
%  must be matched by a \cs{group_end:}, although this does not have
%  to occur within the same function. Indeed, it is often necessary
%  to start a group within one function and finish it within another,
%  for example when seeking to use non-standard category codes.
  これらの関数はそれぞれ定義のためのグループを開始・終了します.
  代入操作はグローバル指定がないかぎりグループ内でローカルに行なわれ
  ます(いくつか例外がありますが,それらは適宜説明します).
  \cs{group_begin:}で開始されたグループは必ず\cs{group_end:}で終了
  される必要があります.ただし,これらは必ずしも同じ関数内で完結して
  いなくても構いません.実際,しばしばある関数でグループを開始して,
  別の関数でグループを終了しなければならないことがあります.例えば
  非標準的なカテゴリーコードを適用したい場合などです.
\end{function}

\begin{function}{\group_insert_after:N}
  \begin{syntax}
    \cs{group_insert_after:N} \meta{token}
  \end{syntax}
%  Adds \meta{token} to the list of \meta{tokens} to be inserted
%  when the current group level ends. The list of \meta{tokens} to be
%  inserted is empty at the beginning of a group: multiple
%  applications of \cs{group_insert_after:N} may be used to build
%  the inserted list one \meta{token} at a time. The current group
%  level may be closed by a \cs{group_end:} function or by a token
%  with category code $2$ (close-group), namely a ^^A{
%  |}| if standard category codes apply.
  \meta{token}を現在の(最も内側の)グループの終了時に挿入される
  トークンのリストに追加します.このトークンのリストはグループの開始
  時点で空リストとなるように作成されます.\cs{group_insert_after:N}を
  複数回使用した場合,リストに格納された各\meta{token}が一度に挿入
  されることになります.現在のグループは\cs{group_end:}関数か,
  カテゴリーコード$2$(グループ終了)のトークン(標準的なカテゴリー
  コード設定下では|}|)によって終了させることができます.
\end{function}

%\section{Control sequences and functions}
\section{制御綴と関数}

%As \TeX{} is a macro language, creating new functions means
%creating macros. At point of use, a function is replaced by
%the replacement text (\enquote{code}) in which each parameter
%in the code (|#1|, |#2|, \emph{etc.}) is replaced the appropriate
%arguments absorbed by the function. In the following, \meta{code}
%is therefore used as a shorthand for \enquote{replacement text}.
\TeX はマクロ言語なので,関数を作ることはマクロを作ることを意味します.
関数を使用すると,その関数は置換テキスト (\enquote{code}) に置き換えられ
ます.このとき置換テキスト内の各パラメータ (|#1|, |#2|, \dots) はそれ
ぞれ関数に与えられた適切な引数に置き換えられます.そのため以下では
\meta{code}は「置換テキスト」のことを指しています.

%Functions which are not \enquote{protected} are fully expanded
%inside an \texttt{x} expansion. In contrast, \enquote{protected}
%functions are not expanded within \texttt{x} expansions.
「保護付き」の関数以外は\code{x}展開では完全展開可能です.一方で
「保護付き」の関数は\code{x}展開においては展開されません.

%\subsection{Defining functions}
\subsection{関数定義}

%Functions can be created with no requirement that they are declared
%first (in contrast to variables, which must always be declared).
%Declaring a function before setting up the code means that the name
%chosen is checked and an error raised if it is already in use.
%The name of a function can be checked at the point of definition using
%the \cs[no-index]{cs_new\ldots} functions: this is recommended for all
%functions which are defined for the first time.
関数は(必ず事前に宣言を必要とする各種変数と異なり)事前に宣言すること
なく作成することができます.関数定義をする前に関数を宣言するということ
には選んだ関数名をチェックし,もし既に使用されている場合にはエラーを出す
という効果があります.\cs[no-index]{cs_new*}系の関数を用いると定義の際に
このチェックを行うことができます.任意の関数を最初に定義する際は,この
方法を使用することが推奨されます.

%There are three ways to define new functions.
%All classes define a function to expand to the substitution text.
%Within the substitution text the actual parameters are substituted
%for the formal parameters (|#1|, |#2|, \ldots).
関数を定義する方法は3種類(クラス)に大別できます.いずれの方法を用いて
も,その関数は置換テキストに展開されるように定義されます.置換テキスト
内のパラメータ (|#1|, |#2|, \dots) は実際の引数に置き換えられます.
%
\begin{description}[style=multiline, leftmargin=3zw]
%  \item[\texttt{new}]
%    Create a new function with the \texttt{new} scope,
%    such as \cs{cs_new:Npn}.  The definition is global and results in
%    an error if it is already defined.
  \item[\code{new}]
    新しい関数は\cs{cs_new:Npn}などの\code{new}系の関数で作成できます.
    これらによる定義はグローバルで,既に定義済みの関数を作成しようと
    するとエラーが発生します.
%
%  \item[\texttt{set}]
%    Create a new function with the \texttt{set} scope,
%    such as \cs{cs_set:Npn}. The definition is restricted to the current
%    \TeX{} group and does not result in an error if the function is already
%    defined.
  \item[\code{set}]
    新しい関数は\cs{cs_set:Npn}などの\code{set}系の関数でも作成できます.
    これらによる定義は現在の(最も内側の)グループ内に限定され,既に
    定義済みの関数であってもエラーは発生しません.
%
%  \item[\texttt{gset}]
%    Create a new function with the \texttt{gset} scope,
%    such as \cs{cs_gset:Npn}. The definition is global and
%    does not result in an error if the function is already defined.
  \item[\code{gset}]
    新しい関数は\cs{cs_gset:Npn}などの\code{gset}系の関数でも作成できます.
    これらによる定義はグローバルで,既に定義済みの関数であってもエラーは
    発生しません.
\end{description}

%Within each set of scope there are different ways to define a function.
%The differences depend on restrictions on the actual parameters and
%the expandability of the resulting function.
上記の各クラスの中でも,関数を定義する方法は多岐にわたります.関数には
実引数や関数の展開可能性を柔軟に設定できます.
%
\begin{description}[style=multiline, leftmargin=6zw]
%  \item[\texttt{nopar}]
%     Create a new function with the \texttt{nopar} restriction,
%     such as \cs{cs_set_nopar:Npn}. The parameter may not contain
%     \cs{par} tokens.
  \item[\code{nopar}]
     \cs{cs_set_nopar:Npn}などで関数を作成すると\code{nopar}制限が付加
     されます.実引数は\cs{par}トークンを含むことができなくなります.
%
%  \item[\texttt{protected}]
%     Create a new function with the \texttt{protected} restriction,
%     such as \cs{cs_set_protected:Npn}. The parameter may contain
%     \cs{par} tokens but the function will not expand within an
%     \texttt{x}-type or \texttt{e}-type expansion.
  \item[\code{protected}]
     \cs{cs_set_protected:Npn}などで関数を作成すると\code{nopar}制限が
     付加されます.実引数は\cs{par}トークンを含むことができますが,
     関数自体は\code{x}展開や\code{e}展開においては展開されなくなります.
\end{description}

%Finally, the functions in
%Subsections~\ref{sec:l3basics:defining-new-function-1}~and
%\ref{sec:l3basics:defining-new-function-2} are primarily meant to define
%\emph{base functions} only. Base functions can only have the following
%argument specifiers:
以下\ref{sec:l3basics:defining-new-function-1}節と\ref{sec:l3basics:%
defining-new-function-2}節で説明する関数は\emph{基本関数}のみを定義する
ことが意図されています.基本関数がもち得る引数指定子は以下のもののみです:
%
\begin{description}[style=multiline, leftmargin=3zw]
%  \item[|N| and |n|] No manipulation.
%  \item[|T| and |F|] Functionally equivalent to |n| (you are actually
%    encouraged to use the family of |\prg_new_conditional:| functions
%    described in Section~\ref{sec:l3prg:new-conditional-functions}).
%  \item[|p| and |w|] These are special cases.
  \item[\code{N}/\code{n}] 操作なし
  \item[\code{T}/\code{F}] 機能的には\code{n}と同様(実践ではここに挙げる
    関数よりも\ref{sec:l3prg:new-conditional-functions}節で説明する
    |\prg_new_conditional:*|系の関数の利用が推奨されます)
  \item[\code{p}/\code{w}] 一部特殊なものもあります
\end{description}

%The |\cs_new:| functions below (and friends) do not stop you from using
%other argument specifiers in your function names, but they do not handle
%expansion for you. You should define the base function and then use
%\cs{cs_generate_variant:Nn} to generate custom variants as described in
%Section~\ref{sec:l3expan:variants-method}.
以下の|\cs_new:*|系の関数(およびその仲間)を用いて上記以外の引数指定子
をもつ関数を定義することもできますが,展開制御が自動的に行なわれることは
ありません.まずは基本関数を定義して,その上で\cs{cs_generate_variant:Nn}%
を用いて独自の変種を生成するのがよいでしょう
(\ref{sec:l3expan:variants-method}節を参照).

%\subsection{Defining new functions using parameter text}
\subsection{パラメータテキストを用いた関数定義}
\label{sec:l3basics:defining-new-function-1}

\begin{function}{\cs_new:Npn, \cs_new:cpn, \cs_new:Npx, \cs_new:cpx}
  \begin{syntax}
    \cs{cs_new:Npn} \meta{function} \meta{parameters} \Arg{code}
  \end{syntax}
%  Creates \meta{function} to expand to \meta{code} as replacement text.
%  Within the \meta{code}, the \meta{parameters} (|#1|, |#2|,
%  \emph{etc.}) will be replaced by those absorbed by the function.
%  The definition is global and an error results if the
%  \meta{function} is already defined.
  \meta{code}を置換テキストとする関数\meta{function}を作成します.
  \meta{code}内では\meta{parameters} (|#1|, |#2|, \dots) はその関数に
  渡された引数に置換されます.この定義はグローバルで,\meta{function}が
  定義済みの場合はエラーが発生します.
\end{function}

\begin{function}
  {
    \cs_new_nopar:Npn, \cs_new_nopar:cpn,
    \cs_new_nopar:Npx, \cs_new_nopar:cpx
  }
  \begin{syntax}
    \cs{cs_new_nopar:Npn} \meta{function} \meta{parameters} \Arg{code}
  \end{syntax}
%  Creates \meta{function} to expand to \meta{code} as replacement text.
%  Within the \meta{code}, the \meta{parameters} (|#1|, |#2|,
%  \emph{etc.}) will be replaced by those absorbed by the function.
%  When the \meta{function} is used the \meta{parameters} absorbed
%  cannot contain \cs{par} tokens. The definition is global and
%  an error results if the \meta{function} is already defined.
  \meta{code}を置換テキストとする関数\meta{function}を作成します.
  \meta{code}内では\meta{parameters} (|#1|, |#2|, \dots) はその関数に
  渡された引数に置換されます.\meta{function}の使用時に渡される引数に
  \cs{par}トークンを含めることはできません.この定義はグローバルで,
  \meta{function}が定義済みの場合はエラーが発生します.
\end{function}

\begin{function}
  {
    \cs_new_protected:Npn, \cs_new_protected:cpn,
    \cs_new_protected:Npx, \cs_new_protected:cpx
  }
  \begin{syntax}
    \cs{cs_new_protected:Npn} \meta{function} \meta{parameters} \Arg{code}
  \end{syntax}
%  Creates \meta{function} to expand to \meta{code} as replacement text.
%  Within the \meta{code}, the \meta{parameters} (|#1|, |#2|,
%  \emph{etc.}) will be replaced by those absorbed by the function.
%  The \meta{function} will not expand within an \texttt{x}-type
%  argument. The definition is global and an error results if the
%  \meta{function} is already defined.
  \meta{code}を置換テキストとする関数\meta{function}を作成します.
  \meta{code}内では\meta{parameters} (|#1|, |#2|, \dots) はその関数に
  渡された引数に置換されます.定義された\meta{function}は\code{x}展開および
  \code{e}展開では展開されません.この定義はグローバルで,\meta{function}が
  定義済みの場合はエラーが発生します.
\end{function}

\begin{function}
  {
    \cs_new_protected_nopar:Npn, \cs_new_protected_nopar:cpn ,
    \cs_new_protected_nopar:Npx, \cs_new_protected_nopar:cpx
  }
  \begin{syntax}
    \cs{cs_new_protected_nopar:Npn} \meta{function} \meta{parameters} \Arg{code}
  \end{syntax}
%  Creates \meta{function} to expand to \meta{code} as replacement text.
%  Within the \meta{code}, the \meta{parameters} (|#1|, |#2|,
%  \emph{etc.}) will be replaced by those absorbed by the function.
%  When the \meta{function} is used the \meta{parameters} absorbed
%  cannot contain \cs{par} tokens. The \meta{function} will not
%  expand within an \texttt{x}-type or \texttt{e}-type argument. The definition is global
%  and an error results if the \meta{function} is already defined.
  \meta{code}を置換テキストとする関数\meta{function}を作成します.
  \meta{code}内では\meta{parameters} (|#1|, |#2|, \dots) はその関数に
  渡された引数に置換されます.\meta{function}の使用時に渡される引数に
  \cs{par}トークンを含めることはできません.定義された\meta{function}は
  \code{x}展開および\code{e}展開では展開されません.この定義はグローバル
  で,\meta{function}が定義済みの場合はエラーが発生します.
\end{function}

\begin{function}{\cs_set:Npn, \cs_set:cpn, \cs_set:Npx, \cs_set:cpx}
  \begin{syntax}
    \cs{cs_set:Npn} \meta{function} \meta{parameters} \Arg{code}
  \end{syntax}
%  Sets \meta{function} to expand to \meta{code} as replacement text.
%  Within the \meta{code}, the \meta{parameters} (|#1|, |#2|,
%  \emph{etc.}) will be replaced by those absorbed by the function.
%  The assignment of a meaning to the \meta{function} is restricted to
%  the current \TeX{} group level.
  関数\meta{function}を\meta{code}に展開されるように設定します.
  \meta{code}内では\meta{parameters} (|#1|, |#2|, \dots) はその関数に
  渡された引数に置換されます.この\meta{function}の意味の設定は,現在の
  最も内側の\TeX グループ内でのみ有効です.
\end{function}

\begin{function}
  {
    \cs_set_nopar:Npn, \cs_set_nopar:cpn,
    \cs_set_nopar:Npx, \cs_set_nopar:cpx
  }
  \begin{syntax}
    \cs{cs_set_nopar:Npn} \meta{function} \meta{parameters} \Arg{code}
  \end{syntax}
%  Sets \meta{function} to expand to \meta{code} as replacement text.
%  Within the \meta{code}, the \meta{parameters} (|#1|, |#2|,
%  \emph{etc.}) will be replaced by those absorbed by the function.
%  When the \meta{function} is used the \meta{parameters} absorbed
%  cannot contain \cs{par} tokens. The assignment of a meaning
%  to the \meta{function} is restricted to the current \TeX{} group
%  level.
  関数\meta{function}を\meta{code}に展開されるように設定します.
  \meta{code}内では\meta{parameters} (|#1|, |#2|, \dots) はその関数に
  渡された引数に置換されます.\meta{function}の使用時に渡される引数に
  \cs{par}トークンを含めることはできません.この\meta{function}の意味
  の設定は,現在の最も内側の\TeX グループ内でのみ有効です.
\end{function}

\begin{function}
  {
    \cs_set_protected:Npn, \cs_set_protected:cpn,
    \cs_set_protected:Npx, \cs_set_protected:cpx
  }
  \begin{syntax}
    \cs{cs_set_protected:Npn} \meta{function} \meta{parameters} \Arg{code}
  \end{syntax}
%  Sets \meta{function} to expand to \meta{code} as replacement text.
%  Within the \meta{code}, the \meta{parameters} (|#1|, |#2|,
%  \emph{etc.}) will be replaced by those absorbed by the function.
%  The assignment of a meaning to the \meta{function} is restricted to
%  the current \TeX{} group level. The \meta{function} will
%  not expand within an \texttt{x}-type or \texttt{e}-type argument.
  関数\meta{function}を\meta{code}に展開されるように設定します.
  \meta{code}内では\meta{parameters} (|#1|, |#2|, \dots) はその関数に
  渡された引数に置換されます.定義された\meta{function}は\code{x}展開
  および\code{e}展開では展開されません.この\meta{function}の意味の
  設定は,現在の最も内側の\TeX グループ内でのみ有効です.
\end{function}

\begin{function}
  {
    \cs_set_protected_nopar:Npn, \cs_set_protected_nopar:cpn ,
    \cs_set_protected_nopar:Npx, \cs_set_protected_nopar:cpx ,
  }
  \begin{syntax}
    \cs{cs_set_protected_nopar:Npn} \meta{function} \meta{parameters} \Arg{code}
  \end{syntax}
%  Sets \meta{function} to expand to \meta{code} as replacement text.
%  Within the \meta{code}, the \meta{parameters} (|#1|, |#2|,
%  \emph{etc.}) will be replaced by those absorbed by the function.
%  When the \meta{function} is used the \meta{parameters} absorbed
%  cannot contain \cs{par} tokens. The assignment of a meaning
%  to the \meta{function} is restricted to the current \TeX{} group
%  level. The \meta{function} will not expand within an
%  \texttt{x}-type or \texttt{e}-type argument.
  関数\meta{function}を\meta{code}に展開されるように設定します.
  \meta{code}内では\meta{parameters} (|#1|, |#2|, \dots) はその関数に
  渡された引数に置換されます.\meta{function}の使用時に渡される引数に
  \cs{par}トークンを含めることはできません.定義された\meta{function}は
  \code{x}展開および\code{e}展開では展開されません.この\meta{function}%
  の意味の設定は,現在の最も内側の\TeX グループ内でのみ有効です.
\end{function}

\begin{function}{\cs_gset:Npn, \cs_gset:cpn, \cs_gset:Npx, \cs_gset:cpx}
  \begin{syntax}
    \cs{cs_gset:Npn} \meta{function} \meta{parameters} \Arg{code}
  \end{syntax}
%  Globally sets \meta{function} to expand to \meta{code} as replacement
%  text. Within the \meta{code}, the \meta{parameters} (|#1|, |#2|,
%  \emph{etc.}) will be replaced by those absorbed by the function.
%  The assignment of a meaning to the \meta{function} is \emph{not}
%  restricted to the current \TeX{} group level: the assignment is
%  global.
  関数\meta{function}を\meta{code}に展開されるようグローバルに設定します.
  \meta{code}内では\meta{parameters} (|#1|, |#2|, \dots) はその関数に
  渡された引数に置換されます.この\meta{function}の意味の設定は,現在の
  \TeX グループに\emph{制限されません}.すなわち,この設定はグローバル
  です.
\end{function}

\begin{function}
  {
    \cs_gset_nopar:Npn, \cs_gset_nopar:cpn,
    \cs_gset_nopar:Npx, \cs_gset_nopar:cpx
  }
  \begin{syntax}
    \cs{cs_gset_nopar:Npn} \meta{function} \meta{parameters} \Arg{code}
  \end{syntax}
%  Globally sets \meta{function} to expand to \meta{code} as replacement
%  text. Within the \meta{code}, the \meta{parameters} (|#1|, |#2|,
%  \emph{etc.}) will be replaced by those absorbed by the function.
%  When the \meta{function} is used the \meta{parameters} absorbed
%  cannot contain \cs{par} tokens. The assignment of a meaning to the
%  \meta{function} is \emph{not} restricted to the current \TeX{}
%  group level: the assignment is global.
  関数\meta{function}を\meta{code}に展開されるようグローバルに設定します.
  \meta{code}内では\meta{parameters} (|#1|, |#2|, \dots) はその関数に
  渡された引数に置換されます.\meta{function}の使用時に渡される引数に
  \cs{par}トークンを含めることはできません.この\meta{function}の意味の
  設定は,現在の\TeX グループに\emph{制限されません}.すなわち,この設定
  はグローバルです.
\end{function}

\begin{function}
  {
    \cs_gset_protected:Npn, \cs_gset_protected:cpn,
    \cs_gset_protected:Npx, \cs_gset_protected:cpx
  }
  \begin{syntax}
    \cs{cs_gset_protected:Npn} \meta{function} \meta{parameters} \Arg{code}
  \end{syntax}
%  Globally sets \meta{function} to expand to \meta{code} as replacement
%  text. Within the \meta{code}, the \meta{parameters} (|#1|, |#2|,
%  \emph{etc.}) will be replaced by those absorbed by the function.
%  The assignment of a meaning to the \meta{function} is \emph{not}
%  restricted to the current \TeX{} group level: the assignment is
%  global. The \meta{function} will not expand within an
%  \texttt{x}-type or \texttt{e}-type argument.
  関数\meta{function}を\meta{code}に展開されるようグローバルに設定します.
  \meta{code}内では\meta{parameters} (|#1|, |#2|, \dots) はその関数に
  渡された引数に置換されます.この\meta{function}の意味の設定は,現在の
  \TeX グループに\emph{制限されません}.すなわち,この設定はグローバル
  です.定義された\meta{function}は\code{x}展開および\code{e}展開では
  展開されません.
\end{function}

\begin{function}
  {
    \cs_gset_protected_nopar:Npn, \cs_gset_protected_nopar:cpn,
    \cs_gset_protected_nopar:Npx, \cs_gset_protected_nopar:cpx
  }
  \begin{syntax}
    \cs{cs_gset_protected_nopar:Npn} \meta{function} \meta{parameters} \Arg{code}
  \end{syntax}
%  Globally sets \meta{function} to expand to \meta{code} as replacement
%  text. Within the \meta{code}, the \meta{parameters} (|#1|, |#2|,
%  \emph{etc.}) will be replaced by those absorbed by the function.
%  When the \meta{function} is used the \meta{parameters} absorbed
%  cannot contain \cs{par} tokens. The assignment of a meaning to the
%  \meta{function} is \emph{not} restricted to the current \TeX{}
%  group level: the assignment is global. The \meta{function} will
%  not expand within an \texttt{x}-type argument.
  関数\meta{function}を\meta{code}に展開されるようグローバルに設定します.
  \meta{code}内では\meta{parameters} (|#1|, |#2|, \dots) はその関数に
  渡された引数に置換されます.\meta{function}の使用時に渡される引数に
  \cs{par}トークンを含めることはできません.この\meta{function}の意味の
  設定は,現在の\TeX グループに\emph{制限されません}.すなわち,この設定は
  グローバルです.定義された\meta{function}は\code{x}展開および\code{e}展開
  では展開されません.
\end{function}

%\subsection{Defining new functions using the signature}
\subsection{シグネチャを用いた関数定義}
\label{sec:l3basics:defining-new-function-2}

\begin{function}
  {
    \cs_new:Nn, \cs_new:cn,
    \cs_new:Nx, \cs_new:cx
  }
  \begin{syntax}
    \cs{cs_new:Nn} \meta{function} \Arg{code}
  \end{syntax}
%  Creates \meta{function} to expand to \meta{code} as replacement text.
%  Within the \meta{code}, the number of \meta{parameters} is detected
%  automatically from the function signature. These \meta{parameters}
%  (|#1|, |#2|, \emph{etc.}) will be replaced by those absorbed by the
%  function. The definition is global and
%  an error results if the \meta{function} is already defined.
  \meta{code}を置換テキストとする関数\meta{function}を作成します.
  この\meta{code}内で使用できる\meta{parameters} (|#1|, |#2|, \dots)
  の数は,関数シグネチャから自動的に特定されます.これらの
  \meta{parameters}はその関数に渡された引数に置換されます.この定義は
  グローバルで,\meta{function}が定義済みの場合はエラーが発生します.
\end{function}

\begin{function}
  {
    \cs_new_nopar:Nn, \cs_new_nopar:cn,
    \cs_new_nopar:Nx, \cs_new_nopar:cx
  }
  \begin{syntax}
    \cs{cs_new_nopar:Nn} \meta{function} \Arg{code}
  \end{syntax}
%  Creates \meta{function} to expand to \meta{code} as replacement text.
%  Within the \meta{code}, the number of \meta{parameters} is detected
%  automatically from the function signature. These \meta{parameters}
%  (|#1|, |#2|, \emph{etc.}) will be replaced by those absorbed by the
%  function.  When the \meta{function} is used the \meta{parameters}
%  absorbed cannot contain \cs{par} tokens. The definition is global and
%  an error results if the \meta{function} is already defined.
  \meta{code}を置換テキストとする関数\meta{function}を作成します.
  この\meta{code}内で使用できる\meta{parameters} (|#1|, |#2|, \dots)
  の数は,関数シグネチャから自動的に特定されます.これらの
  \meta{parameters}はその関数に渡された引数に置換されます.
  \meta{function}の使用時に渡される引数に\cs{par}トークンを含めることは
  できません.この定義はグローバルで,\meta{function}が定義済みの場合は
  エラーが発生します.
\end{function}

\begin{function}
  {
    \cs_new_protected:Nn, \cs_new_protected:cn,
    \cs_new_protected:Nx, \cs_new_protected:cx
  }
  \begin{syntax}
    \cs{cs_new_protected:Nn} \meta{function} \Arg{code}
  \end{syntax}
%  Creates \meta{function} to expand to \meta{code} as replacement text.
%  Within the \meta{code}, the number of \meta{parameters} is detected
%  automatically from the function signature. These \meta{parameters}
%  (|#1|, |#2|, \emph{etc.}) will be replaced by those absorbed by the
%  function. The \meta{function} will not expand within an \texttt{x}-type
%  argument. The definition is global and
%  an error results if the \meta{function} is already defined.
  \meta{code}を置換テキストとする関数\meta{function}を作成します.
  この\meta{code}内で使用できる\meta{parameters} (|#1|, |#2|, \dots)
  の数は,関数シグネチャから自動的に特定されます.これらの
  \meta{parameters}はその関数に渡された引数に置換されます.定義された
  \meta{function}は\code{x}展開および\code{e}展開では展開されません.
  この定義はグローバルで,\meta{function}が定義済みの場合はエラーが
  発生します.
\end{function}

\begin{function}
  {
    \cs_new_protected_nopar:Nn, \cs_new_protected_nopar:cn,
    \cs_new_protected_nopar:Nx, \cs_new_protected_nopar:cx
  }
  \begin{syntax}
    \cs{cs_new_protected_nopar:Nn} \meta{function} \Arg{code}
  \end{syntax}
%  Creates \meta{function} to expand to \meta{code} as replacement text.
%  Within the \meta{code}, the number of \meta{parameters} is detected
%  automatically from the function signature. These \meta{parameters}
%  (|#1|, |#2|, \emph{etc.}) will be replaced by those absorbed by the
%  function.  When the \meta{function} is used the \meta{parameters}
%  absorbed cannot contain \cs{par} tokens. The \meta{function} will not
%  expand within an \texttt{x}-type or \texttt{e}-type argument. The definition is global and
%  an error results if the \meta{function} is already defined.
  \meta{code}を置換テキストとする関数\meta{function}を作成します.
  この\meta{code}内で使用できる\meta{parameters} (|#1|, |#2|, \dots)
  の数は,関数シグネチャから自動的に特定されます.これらの
  \meta{parameters}はその関数に渡された引数に置換されます.
  \meta{function}の使用時に渡される引数に\cs{par}トークンを含めることは
  できません.定義された\meta{function}は\code{x}展開および\code{e}展開
  では展開されません.この定義はグローバルで,\meta{function}が定義済み
  の場合はエラーが発生します.
\end{function}

\begin{function}
  {
    \cs_set:Nn, \cs_set:cn,
    \cs_set:Nx, \cs_set:cx
  }
  \begin{syntax}
    \cs{cs_set:Nn} \meta{function} \Arg{code}
  \end{syntax}
%  Sets \meta{function} to expand to \meta{code} as replacement text.
%  Within the \meta{code}, the number of \meta{parameters} is detected
%  automatically from the function signature. These \meta{parameters}
%  (|#1|, |#2|, \emph{etc.}) will be replaced by those absorbed by the
%  function.
%  The assignment of a meaning to the \meta{function} is restricted to
%  the current \TeX{} group level.
  関数\meta{function}を\meta{code}に展開されるように設定します.
  この\meta{code}内で使用できる\meta{parameters} (|#1|, |#2|, \dots)
  の数は,関数シグネチャから自動的に特定されます.これらの
  \meta{parameters}はその関数に渡された引数に置換されます.この定義は
  グローバルで,\meta{function}が定義済みの場合はエラーが発生します.
\end{function}

\begin{function}
  {
    \cs_set_nopar:Nn, \cs_set_nopar:cn,
    \cs_set_nopar:Nx, \cs_set_nopar:cx
  }
  \begin{syntax}
    \cs{cs_set_nopar:Nn} \meta{function} \Arg{code}
  \end{syntax}
%  Sets \meta{function} to expand to \meta{code} as replacement text.
%  Within the \meta{code}, the number of \meta{parameters} is detected
%  automatically from the function signature. These \meta{parameters}
%  (|#1|, |#2|, \emph{etc.}) will be replaced by those absorbed by the
%  function.  When the \meta{function} is used the \meta{parameters}
%  absorbed cannot contain \cs{par} tokens.
%  The assignment of a meaning to the \meta{function} is restricted to
%  the current \TeX{} group level.
  関数\meta{function}を\meta{code}に展開されるように設定します.
  この\meta{code}内で使用できる\meta{parameters} (|#1|, |#2|, \dots)
  の数は,関数シグネチャから自動的に特定されます.これらの
  \meta{parameters}はその関数に渡された引数に置換されます.
  \meta{function}の使用時に渡される引数に\cs{par}トークンを含めることは
  できません.この\meta{function}の意味の設定は,現在の最も内側の\TeX
  グループ内でのみ有効です.
\end{function}

\begin{function}
  {
    \cs_set_protected:Nn, \cs_set_protected:cn,
    \cs_set_protected:Nx, \cs_set_protected:cx
  }
  \begin{syntax}
    \cs{cs_set_protected:Nn} \meta{function} \Arg{code}
  \end{syntax}
%  Sets \meta{function} to expand to \meta{code} as replacement text.
%  Within the \meta{code}, the number of \meta{parameters} is detected
%  automatically from the function signature. These \meta{parameters}
%  (|#1|, |#2|, \emph{etc.}) will be replaced by those absorbed by the
%  function. The \meta{function} will not expand within an \texttt{x}-type
%  argument.
%  The assignment of a meaning to the \meta{function} is restricted to
%  the current \TeX{} group level.
  関数\meta{function}を\meta{code}に展開されるように設定します.
  この\meta{code}内で使用できる\meta{parameters} (|#1|, |#2|, \dots)
  の数は,関数シグネチャから自動的に特定されます.これらの
  \meta{parameters}はその関数に渡された引数に置換されます.定義された
  \meta{function}は\code{x}展開および\code{e}展開では展開されません.
  この\meta{function}の意味の設定は,現在の最も内側の\TeX グループ内
  でのみ有効です.
\end{function}

\begin{function}
  {
    \cs_set_protected_nopar:Nn, \cs_set_protected_nopar:cn,
    \cs_set_protected_nopar:Nx, \cs_set_protected_nopar:cx
  }
  \begin{syntax}
    \cs{cs_set_protected_nopar:Nn} \meta{function} \Arg{code}
  \end{syntax}
%  Sets \meta{function} to expand to \meta{code} as replacement text.
%  Within the \meta{code}, the number of \meta{parameters} is detected
%  automatically from the function signature. These \meta{parameters}
%  (|#1|, |#2|, \emph{etc.}) will be replaced by those absorbed by the
%  function.  When the \meta{function} is used the \meta{parameters}
%  absorbed cannot contain \cs{par} tokens. The \meta{function} will not
%  expand within an \texttt{x}-type or \texttt{e}-type argument.
%  The assignment of a meaning to the \meta{function} is restricted to
%  the current \TeX{} group level.
  関数\meta{function}を\meta{code}に展開されるように設定します.
  この\meta{code}内で使用できる\meta{parameters} (|#1|, |#2|, \dots)
  の数は,関数シグネチャから自動的に特定されます.これらの
  \meta{parameters}はその関数に渡された引数に置換されます.
  \meta{function}の使用時に渡される引数に\cs{par}トークンを含めることは
  できません.定義された\meta{function}は\code{x}展開および\code{e}展開
  では展開されません.この\meta{function}の意味の設定は,現在の最も
  内側の\TeX グループ内でのみ有効です.
\end{function}

\begin{function}
  {
    \cs_gset:Nn, \cs_gset:cn,
    \cs_gset:Nx, \cs_gset:cx
  }
  \begin{syntax}
    \cs{cs_gset:Nn} \meta{function} \Arg{code}
  \end{syntax}
%  Sets \meta{function} to expand to \meta{code} as replacement text.
%  Within the \meta{code}, the number of \meta{parameters} is detected
%  automatically from the function signature. These \meta{parameters}
%  (|#1|, |#2|, \emph{etc.}) will be replaced by those absorbed by the
%  function.
%  The assignment of a meaning to the \meta{function} is  global.
  関数\meta{function}を\meta{code}に展開されるようグローバルに設定します.
  この\meta{code}内で使用できる\meta{parameters} (|#1|, |#2|, \dots)
  の数は,関数シグネチャから自動的に特定されます.これらの
  \meta{parameters}はその関数に渡された引数に置換されます.この設定は
  グローバルです.
\end{function}

\begin{function}
  {
    \cs_gset_nopar:Nn, \cs_gset_nopar:cn,
    \cs_gset_nopar:Nx, \cs_gset_nopar:cx
  }
  \begin{syntax}
    \cs{cs_gset_nopar:Nn} \meta{function} \Arg{code}
  \end{syntax}
%  Sets \meta{function} to expand to \meta{code} as replacement text.
%  Within the \meta{code}, the number of \meta{parameters} is detected
%  automatically from the function signature. These \meta{parameters}
%  (|#1|, |#2|, \emph{etc.}) will be replaced by those absorbed by the
%  function.  When the \meta{function} is used the \meta{parameters}
%  absorbed cannot contain \cs{par} tokens.
%  The assignment of a meaning to the \meta{function} is global.
  関数\meta{function}を\meta{code}に展開されるようグローバルに設定します.
  この\meta{code}内で使用できる\meta{parameters} (|#1|, |#2|, \dots)
  の数は,関数シグネチャから自動的に特定されます.これらの
  \meta{parameters}はその関数に渡された引数に置換されます.
  \meta{function}の使用時に渡される引数に\cs{par}トークンを含めることは
  できません.この設定はグローバルです.
\end{function}

\begin{function}
  {
    \cs_gset_protected:Nn, \cs_gset_protected:cn,
    \cs_gset_protected:Nx, \cs_gset_protected:cx
  }
  \begin{syntax}
    \cs{cs_gset_protected:Nn} \meta{function} \Arg{code}
  \end{syntax}
%  Sets \meta{function} to expand to \meta{code} as replacement text.
%  Within the \meta{code}, the number of \meta{parameters} is detected
%  automatically from the function signature. These \meta{parameters}
%  (|#1|, |#2|, \emph{etc.}) will be replaced by those absorbed by the
%  function. The \meta{function} will not expand within an \texttt{x}-type
%  argument.
%  The assignment of a meaning to the \meta{function} is  global.
  関数\meta{function}を\meta{code}に展開されるようグローバルに設定します.
  この\meta{code}内で使用できる\meta{parameters} (|#1|, |#2|, \dots)
  の数は,関数シグネチャから自動的に特定されます.これらの
  \meta{parameters}はその関数に渡された引数に置換されます.定義された
  \meta{function}は\code{x}展開および\code{e}展開では展開されません.
  この設定はグローバルです.
\end{function}

\begin{function}
  {
    \cs_gset_protected_nopar:Nn, \cs_gset_protected_nopar:cn,
    \cs_gset_protected_nopar:Nx, \cs_gset_protected_nopar:cx
  }
  \begin{syntax}
    \cs{cs_gset_protected_nopar:Nn} \meta{function} \Arg{code}
  \end{syntax}
%  Sets \meta{function} to expand to \meta{code} as replacement text.
%  Within the \meta{code}, the number of \meta{parameters} is detected
%  automatically from the function signature. These \meta{parameters}
%  (|#1|, |#2|, \emph{etc.}) will be replaced by those absorbed by the
%  function.  When the \meta{function} is used the \meta{parameters}
%  absorbed cannot contain \cs{par} tokens. The \meta{function} will not
%  expand within an \texttt{x}-type or \texttt{e}-type argument.
%  The assignment of a meaning to the \meta{function} is global.
  関数\meta{function}を\meta{code}に展開されるようグローバルに設定します.
  この\meta{code}内で使用できる\meta{parameters} (|#1|, |#2|, \dots)
  の数は,関数シグネチャから自動的に特定されます.これらの
  \meta{parameters}はその関数に渡された引数に置換されます.
  \meta{function}の使用時に渡される引数に\cs{par}トークンを含めることは
  できません.定義された\meta{function}は\code{x}展開および\code{e}展開
  では展開されません.この設定はグローバルです.
\end{function}

\begin{function}[updated = 2012-01-14]
  {
    \cs_generate_from_arg_count:NNnn,
    \cs_generate_from_arg_count:cNnn,
    \cs_generate_from_arg_count:Ncnn
  }
  \begin{syntax}
    \cs{cs_generate_from_arg_count:NNnn} \meta{function} \meta{creator} \Arg{number} \Arg{code}
  \end{syntax}
%  Uses the \meta{creator} function (which should have signature
%  |Npn|, for example \cs{cs_new:Npn}) to define a \meta{function}
%  which takes \meta{number} arguments and has \meta{code} as
%  replacement text. The \meta{number} of arguments is an integer expression,
%  evaluated as detailed for \cs{int_eval:n}.
  関数を作成する関数\meta{creator}を用いて,関数\meta{function}を引数を
  \meta{number}個取り,置換テキストを\meta{code}とするものとして定義
  します.\meta{creator}は\cs{cs_new:Npn}などシグネチャ|Npn|をもつ関数
  である必要があります.引数の数\meta{number}は整数表現で,\cs{int_eval:n}%
  関数のところで説明されるように評価されます.
\end{function}

%\subsection{Copying control sequences}
\subsection{制御綴の複製}

%Control sequences (not just functions as defined above) can
%be set to have the same meaning using the functions described
%here. Making two control sequences equivalent means that the
%second control sequence is a \emph{copy} of the first (rather than
%a pointer to it). Thus the old and new control sequence are not
%tied together: changes to one are not reflected in the other.
制御綴は(上記の方法で定義された関数以外のものも含めて)ここで紹介する
関数を用いることにより,別の制御綴と同一の意味を持つように設定することが
可能です.2つの制御綴を同一化するということは,一方の制御綴をもう一方に
(ポインタを張るのではなく)コピーするということを意味します.したがって
元の制御綴と新しい制御綴は互いに関係を持ちません.つまり,一方の意味を
変更したとしても,もう一方には影響を及ぼさないということです.

%In the following text \enquote{cs} is used as an abbreviation for
%\enquote{control sequence}.
以下の説明では略語\enquote{cs}を制御綴 (control sequence) の意味で
用います.

\begin{function}
  {\cs_new_eq:NN, \cs_new_eq:Nc, \cs_new_eq:cN, \cs_new_eq:cc}
  \begin{syntax}
    \cs{cs_new_eq:NN} \meta{cs_1} \meta{cs_2}
    \cs{cs_new_eq:NN} \meta{cs_1} \meta{token}
  \end{syntax}
%  Globally creates \meta{control sequence_1} and sets it to have the same
%  meaning as \meta{control sequence_2} or <token>.
%  The second control sequence may
%  subsequently be altered without affecting the copy.
  \meta{cs_1}をグローバルに作成し,その意味を\meta{cs_2}または\meta{token}と
  同じものに設定します.\meta{cs_2}の意味がその後に変更されたとしても,
  コピーには影響を与えません.
\end{function}

\begin{function}
  {\cs_set_eq:NN, \cs_set_eq:Nc, \cs_set_eq:cN, \cs_set_eq:cc}
  \begin{syntax}
    \cs{cs_set_eq:NN} \meta{cs_1} \meta{cs_2}
    \cs{cs_set_eq:NN} \meta{cs_1} \meta{token}
  \end{syntax}
%  Sets \meta{control sequence_1} to have the same meaning as
%  \meta{control sequence_2} (or <token>).
%  The second control sequence may subsequently be
%  altered without affecting the copy. The assignment of a meaning
%  to the \meta{control sequence_1} is restricted to the current
%  \TeX{} group level.
  \meta{cs_1}を\meta{cs_2}または\meta{token}と同じ意味を持つように設定します.
  \meta{cs_2}の意味がその後に変更されたとしても,コピーには影響を与えません.
  \meta{cs_1}への意味の代入は現在の\TeX グループの中でのみ有効です.
\end{function}

\begin{function}
  {\cs_gset_eq:NN, \cs_gset_eq:Nc, \cs_gset_eq:cN, \cs_gset_eq:cc}
  \begin{syntax}
    \cs{cs_gset_eq:NN} \meta{cs_1} \meta{cs_2}
    \cs{cs_gset_eq:NN} \meta{cs_1} \meta{token}
  \end{syntax}
%  Globally sets \meta{control sequence_1} to have the same meaning as
%  \meta{control sequence_2} (or <token>).
%  The second control sequence may subsequently be
%  altered without affecting the copy. The assignment of a meaning to
%  the \meta{control sequence_1} is \emph{not} restricted to the current
%  \TeX{} group level: the assignment is global.
  \meta{cs_1}を\meta{cs_2}または\meta{token}と同じ意味を持つようにグローバル
  に設定します.\meta{cs_2}の意味がその後に変更されたとしても,コピーには
  影響を与えません.\meta{cs_1}への意味の代入は現在の\TeX グループの中に
  制限されません.この代入はグローバルです.
\end{function}

%\subsection{Deleting control sequences}
\subsection{制御綴の削除}

%There are occasions where control sequences need to be deleted.
%This is handled in a very simple manner.
時として制御綴を削除したい場合があります.次の関数は,この需要をとても
シンプルに満たします.

\begin{function}[updated = 2011-09-15]{\cs_undefine:N, \cs_undefine:c}
  \begin{syntax}
    \cs{cs_undefine:N} \meta{control sequence}
  \end{syntax}
%  Sets \meta{control sequence} to be globally undefined.
  制御綴\meta{control sequence}をグローバルに未定義にします.
\end{function}

%\subsection{Showing control sequences}
\subsection{制御綴の表示}

\begin{function}[EXP, updated = 2011-12-22]{\cs_meaning:N, \cs_meaning:c}
  \begin{syntax}
    \cs{cs_meaning:N} \meta{control sequence}
  \end{syntax}
%  This function expands to the \emph{meaning} of the \meta{control sequence}
%  control sequence. For a macro, this includes the \meta{replacement text}.
  この関数は制御綴\meta{control sequence}の「意味」へと展開されます.
  マクロについては,その「意味」には置換テキストも含まれます.
  %
  \begin{texnote}
  %  This is \TeX{}'s \tn{meaning} primitive.
  %  For tokens that are not control sequences, it is more logical to
  %  use \cs{token_to_meaning:N}.
  %  The \texttt{c} variant correctly reports undefined arguments.
    これは\TeX の\tn{meaning}プリミティブです.制御綴以外のトークンに対しては
    \cs{token_to_meaning:N}関数を用いる方が論理的です.\code{c}変種では,
    未定義の制御綴名についてはその旨を正しく報告します.
  \end{texnote}
\end{function}

\begin{function}[updated = 2017-02-14]{\cs_show:N, \cs_show:c}
  \begin{syntax}
    \cs{cs_show:N} \meta{control sequence}
  \end{syntax}
%  Displays the definition of the \meta{control sequence} on the
%  terminal.
  制御綴\meta{control sequence}の定義をターミナルに表示します.
  %
  \begin{texnote}
  %  This is similar to the \TeX{} primitive \tn{show}, wrapped to a
  %  fixed number of characters per line.
    これは\TeX プリミティブ\tn{show}と類似していますが,行における文字数が
    固定されるようラップされています.
  \end{texnote}
\end{function}

\begin{function}[added = 2014-08-22, updated = 2017-02-14]{\cs_log:N, \cs_log:c}
  \begin{syntax}
    \cs{cs_log:N} \meta{control sequence}
  \end{syntax}
%  Writes the definition of the \meta{control sequence} in the log
%  file.  See also \cs{cs_show:N} which displays the result in the
%  terminal.
  制御綴\meta{control sequence}の定義をログファイルに書き込みます.
  ターミナルに定義を表示する\cs{cs_show:N}も参照してください.
\end{function}

%\subsection{Converting to and from control sequences}
\subsection{制御綴と文字列の相互変換}

\begin{function}[EXP]{\use:c}
  \begin{syntax}
    \cs{use:c} \Arg{control sequence name}
  \end{syntax}
%  Expands the \meta{control sequence name} until only characters
%  remain, and then converts this into a control sequence. This process
%  requires two expansions.  As in other \texttt{c}-type arguments the
%  \meta{control sequence name} must, when fully expanded, consist of
%  character tokens, typically a mixture of category code $10$ (space),
%  $11$ (letter) and $12$ (other).
  制御綴名\meta{control sequence name}を文字のみが残る状態まで展開し,
  その後その文字列を制御綴へと変換します.この処理には2回の展開が必要
  です.他の\code{c}タイプの引数と同様に\meta{control sequence name}は
  完全展開された状態では必ず文字トークンで構成されている必要があります.
  つまり,典型的にはカテゴリーコードが$10$(空白),$11$(文字),$12$
  (その他の文字)のいずれかをもつトークンの混合になります.
%
  \begin{texnote}
%    Protected macros that appear in a \texttt{c}-type argument are
%    expanded despite being protected; \cs{exp_not:n} also has no
%    effect.  An internal error occurs if non-characters or active
%    characters remain after full expansion, as the conversion to a
%    control sequence is not possible.
    \code{c}タイプの引数の中では保護付きのマクロも関係なく展開されます.
    \cs{exp_not:n}も同様に無効です.完全展開の後も文字以外のトークンや
    アクティブ文字が残る場合,制御綴への変換ができないため内部エラーが
    発生します.
  \end{texnote}
\end{function}

%As an example of the \cs{use:c} function, both
\cs{use:c}の使用例を挙げます.
%
  \begin{verbatim}
    \use:c { a b c }
  \end{verbatim}
%
%  and
  および
%
  \begin{verbatim}
    \tl_new:N  \l_my_tl
    \tl_set:Nn \l_my_tl { a b c }
    \use:c { \tl_use:N \l_my_tl }
  \end{verbatim}
%
%  would be equivalent to
  はいずれも,\cs{use:c}を2回展開した後に以下と同義です:
%
  \begin{verbatim}
    \abc
  \end{verbatim}
%  after two expansions of \cs{use:c}.

\begin{function}[noTF, EXP, added = 2012-11-10]
  {\cs_if_exist_use:N, \cs_if_exist_use:c}
  \begin{syntax}
    \cs{cs_if_exist_use:N} \meta{control sequence}
    \cs{cs_if_exist_use:NTF} \meta{control sequence} \Arg{true code} \Arg{false code}
  \end{syntax}
%  Tests whether the \meta{control sequence} is currently defined
%  according to the conditional \cs{cs_if_exist:NTF}
%  (whether as a function or another control sequence type), and if it
%  is inserts the  \meta{control sequence} into the input stream followed
%  by the \meta{true code}.  Otherwise the \meta{false code} is used.
  \meta{control sequence}が現時点で定義済みか否かを条件分岐
  \cs{cs_if_exist:NTF}に基づいてテストし(関数かその他の制御綴かは
  問われません),定義済みの場合は\meta{control sequence}とその直後に
  \meta{true code}を入力ストリームに挿入します.一方,未定義の場合は
  \meta{false code}が使われます.
\end{function}

\begin{function}[EXP]{\cs:w, \cs_end:}
  \begin{syntax}
    \cs{cs:w} \meta{control sequence name} \cs{cs_end:}
  \end{syntax}
%  Converts the given \meta{control sequence name} into a single
%  control sequence token. This process requires one expansion.
%  The content for \meta{control sequence name} may be literal
%  material or from other expandable functions. The
%  \meta{control sequence name} must, when fully expanded, consist
%  of character tokens which are not active: typically
%  of category code $10$ (space), $11$ (letter)
%  or $12$ (other), or a mixture of these.
  与えられた\meta{control sequence name}を1つの制御綴トークンに変換
  します.この処理には展開が必要です.\meta{control sequence name}には
  単純な文字列か他の展開可能関数を含めることができます.\meta{control
  sequence name}は完全展開ののちアクティブでない文字トークンのみで構成
  されている必要があります.典型的には,カテゴリーコード$10$(空白),
  $11$(文字),$12$(その他)およびこれらの混合ということになります.
%
  \begin{texnote}
%    These are the \TeX{} primitives \tn{csname} and \tn{endcsname}.
    これらは\TeX プリミティブ\tn{csname}と\tn{endcsname}です.
  \end{texnote}
\end{function}

\cs{cs:w}と\cs{cs_end:}の利用例を挙げます.
%As an example of the \cs{cs:w} and \cs{cs_end:} functions, both
  \begin{verbatim}
    \cs:w a b c \cs_end:
  \end{verbatim}
  および
%  and
  \begin{verbatim}
    \tl_new:N  \l_my_tl
    \tl_set:Nn \l_my_tl { a b c }
    \cs:w \tl_use:N \l_my_tl \cs_end:
  \end{verbatim}
  はいずれも1回展開の後
%  would be equivalent to
  \begin{verbatim}
    \abc
  \end{verbatim}
  と等価になります.
%  after one expansion of \cs{cs:w}.

\begin{function}[EXP]{\cs_to_str:N}
  \begin{syntax}
    \cs{cs_to_str:N} \meta{control sequence}
  \end{syntax}
  与えられた\meta{control sequence}を,空白(カテゴリーコード10)を除きすべて
  カテゴリーコード12(その他)の文字の列に変換します.\cs{token_to_str:N}とは
  異なり,変換決kは現在のエスケープ文字を\emph{含みません}.この関数を完全に
  展開するにはちょうど2回の展開が必要です.つまり\meta{control sequence}を文字
  の列に変換して入力ストリームに送るには,1回の\code{x}タイプまたは\code{e}%
  タイプの展開もしくは2回の\code{o}タイプによる展開が必要となります.ほとんどの
  場合,\code{f}展開を1回行っても問題ありませんが,変換結果の冒頭の空白文字が
  1つ消えてしまいます.
%  Converts the given \meta{control sequence} into a series of
%  characters with category code $12$ (other), except spaces,
%  of category code $10$. The result does \emph{not} include
%  the current escape token, contrarily to \cs{token_to_str:N}.
%  Full expansion of this function requires exactly $2$ expansion
%  steps, and so an \texttt{x}-type or \texttt{e}-type expansion, or two
%  \texttt{o}-type expansions are required to
%  convert the \meta{control sequence} to a sequence of characters
%  in the input stream. In most cases, an \texttt{f}-expansion
%  is correct as well, but this loses a space at the start
%  of the result.
\end{function}

\section{Analysing control sequences}

\begin{function}[EXP, added = 2018-04-06]{\cs_split_function:N}
  \begin{syntax}
    \cs{cs_split_function:N} \meta{function}
  \end{syntax}
  Splits the \meta{function} into the \meta{name} (\emph{i.e.}~the part
  before the colon) and the \meta{signature} (\emph{i.e.}~after the colon).
  This information is then placed in the input stream
  in three parts: the \meta{name}, the
  \meta{signature} and a logic token indicating if a colon was found
  (to differentiate variables from function names). The \meta{name}
  does not include the escape character, and both the \meta{name} and
  \meta{signature} are made up of tokens with category code $12$
  (other).
\end{function}

The next three functions decompose \TeX{} macros into their
constituent parts: if the \meta{token} passed is not a macro then no
decomposition can occur. In the latter case, all three functions leave
\cs{scan_stop:} in the input stream.

\begin{function}[EXP, added = 2019-02-27]{\cs_prefix_spec:N}
  \begin{syntax}
    \cs{cs_prefix_spec:N} \meta{token}
  \end{syntax}
  If the \meta{token} is a macro, this function leaves the applicable
  \TeX{} prefixes in input stream as a string of tokens of category
  code $12$ (with spaces having category code $10$). Thus for example
  \begin{verbatim}
    \cs_set:Npn \next:nn #1#2 { x #1~y #2 }
    \cs_prefix_spec:N \next:nn
  \end{verbatim}
  leaves |\long| in the input stream. If the \meta{token} is
  not a macro then \cs{scan_stop:} is left in the input stream.
  \begin{texnote}
    The prefix can be empty, |\long|, |\protected| or
    |\protected\long| with backslash replaced by the current escape
    character.
  \end{texnote}
\end{function}

\begin{function}[EXP, added = 2019-02-27]{\cs_argument_spec:N}
  \begin{syntax}
    \cs{cs_argument_spec:N} \meta{token}
  \end{syntax}
  If the \meta{token} is a macro, this function leaves the primitive
  \TeX{} argument specification in input stream as a string of
  character tokens of category code $12$ (with spaces having category
  code $10$). Thus for example
  \begin{verbatim}
    \cs_set:Npn \next:nn #1#2 { x #1 y #2 }
    \cs_argument_spec:N \next:nn
  \end{verbatim}
  leaves |#1#2| in the input stream. If the \meta{token} is
  not a macro then \cs{scan_stop:} is left in the input stream.
  \begin{texnote}
    If the argument specification contains the string |->|, then the
    function produces incorrect results.
  \end{texnote}
\end{function}

\begin{function}[EXP, added = 2019-02-27]{\cs_replacement_spec:N}
  \begin{syntax}
    \cs{cs_replacement_spec:N} \meta{token}
  \end{syntax}
  If the \meta{token} is a macro, this function leaves the replacement
  text in input stream as a string of character tokens of category
  code $12$ (with spaces having category code $10$). Thus for example
  \begin{verbatim}
    \cs_set:Npn \next:nn #1#2 { x #1~y #2 }
    \cs_replacement_spec:N \next:nn
  \end{verbatim}
  leaves \verb*|x#1 y#2| in the input stream. If the \meta{token} is
  not a macro then \cs{scan_stop:} is left in the input stream.
  \begin{texnote}
    If the argument specification contains the string |->|, then the
    function produces incorrect results.
  \end{texnote}
\end{function}

\section{Using or removing tokens and arguments}

Tokens in the input can be read and used or read and discarded.
If one or more tokens are wrapped in braces then when absorbing them
the outer set is removed. At the same time, the category code
of each token is set when the token is read by a function (if it
is read more than once, the category code is determined by
the situation in force when first function absorbs the token).

\begin{function}[EXP]{\use:n, \use:nn, \use:nnn, \use:nnnn}
  \begin{syntax}
    \cs{use:n}    \Arg{group_1}
    \cs{use:nn}   \Arg{group_1} \Arg{group_2}
    \cs{use:nnn}  \Arg{group_1} \Arg{group_2} \Arg{group_3}
    \cs{use:nnnn} \Arg{group_1} \Arg{group_2} \Arg{group_3} \Arg{group_4}
  \end{syntax}
  As illustrated, these functions absorb between one and four
  arguments, as indicated by the argument specifier. The braces
  surrounding each argument are removed and the remaining tokens are
  left in the input stream. The category code of these tokens is
  also fixed by this process (if it has not already been by some
  other absorption). All of these functions require only a single
  expansion to operate, so that one expansion of
  \begin{verbatim}
    \use:nn { abc } { { def } }
  \end{verbatim}
  results in the input stream containing
  \begin{verbatim}
    abc { def }
  \end{verbatim}
  \emph{i.e.} only the outer braces are removed.
  \begin{texnote}
    The \cs{use:n} function is equivalent to \LaTeXe{}'s \tn{@firstofone}.
  \end{texnote}
\end{function}

\begin{function}[EXP]{\use_i:nn, \use_ii:nn}
   \begin{syntax}
    \cs{use_i:nn} \Arg{arg_1} \Arg{arg_2}
  \end{syntax}
  These functions absorb two arguments from the input stream. The function
  \cs{use_i:nn} discards the second argument, and leaves the content
  of the first argument in the input stream. \cs{use_ii:nn} discards
  the first argument and leaves the content of the second argument in
  the input stream.
  The category code
  of these tokens is also fixed (if it has not already been by
  some other absorption). A single expansion is needed for the
  functions to take effect.
\end{function}

\begin{texnote}
  These are equivalent to \LaTeXe{}'s \tn{@firstoftwo} and
  \tn{@secondoftwo}.
\end{texnote}

\begin{function}[EXP]{\use_i:nnn, \use_ii:nnn, \use_iii:nnn}
  \begin{syntax}
    \cs{use_i:nnn} \Arg{arg_1} \Arg{arg_2} \Arg{arg_3}
  \end{syntax}
  These functions absorb three arguments from the input stream. The
  function \cs{use_i:nnn} discards the second and third arguments, and
  leaves the content of the first argument in the input stream.
  \cs{use_ii:nnn} and \cs{use_iii:nnn} work similarly, leaving the
  content of second or third arguments in the input stream, respectively.
  The category code
  of these tokens is also fixed (if it has not already been by
  some other absorption). A single expansion is needed for the
  functions to take effect.
\end{function}

\begin{function}[EXP]
  {\use_i:nnnn, \use_ii:nnnn, \use_iii:nnnn, \use_iv:nnnn}
  \begin{syntax}
    \cs{use_i:nnnn} \Arg{arg_1} \Arg{arg_2} \Arg{arg_3} \Arg{arg_4}
  \end{syntax}
  These functions absorb four arguments from the input stream. The
  function \cs{use_i:nnnn} discards the second, third and fourth
  arguments, and leaves the content of the first argument in the input
  stream.
  \cs{use_ii:nnnn}, \cs{use_iii:nnnn} and \cs{use_iv:nnnn} work similarly,
  leaving the content of second, third or fourth arguments in the input
  stream, respectively. The category code
  of these tokens is also fixed (if it has not already been by
  some other absorption). A single expansion is needed for the
  functions to take effect.
\end{function}

\begin{function}[EXP]{\use_i_ii:nnn}
  \begin{syntax}
    \cs{use_i_ii:nnn} \Arg{arg_1} \Arg{arg_2} \Arg{arg_3}
  \end{syntax}
  This function absorbs three arguments and leaves the content of the
  first and second in the input stream. The category code of
  these tokens is also fixed (if it has not already been by
  some other absorption). A single expansion is needed for the
  function to take effect. An example:
  \begin{verbatim}
    \use_i_ii:nnn { abc } { { def } } { ghi }
  \end{verbatim}
  results in the input stream containing
  \begin{verbatim}
    abc { def }
  \end{verbatim}
  \emph{i.e.} the outer braces are removed and the third group
  is removed.
\end{function}

\begin{function}[EXP, added = 2019-06-02]{\use_ii_i:nn}
  \begin{syntax}
    \cs{use_ii_i:nn} \Arg{arg_1} \Arg{arg_2}
  \end{syntax}
  This function absorbs two arguments and leaves the content of the
  second and first in the input stream. The category code of
  these tokens is also fixed (if it has not already been by
  some other absorption). A single expansion is needed for the
  function to take effect.
\end{function}

\begin{function}[EXP]
  {
    \use_none:n         ,
    \use_none:nn        ,
    \use_none:nnn       ,
    \use_none:nnnn      ,
    \use_none:nnnnn     ,
    \use_none:nnnnnn    ,
    \use_none:nnnnnnn   ,
    \use_none:nnnnnnnn  ,
    \use_none:nnnnnnnnn
  }
  \begin{syntax}
    \cs{use_none:n} \Arg{group_1}
  \end{syntax}
  These functions absorb between one and nine groups from the
  input stream, leaving nothing on the resulting input stream.
  These functions work after a single expansion. One or more of the
  \texttt{n} arguments may be an unbraced single token
  (\emph{i.e.}~an \texttt{N} argument).
  \begin{texnote}
    These are equivalent to \LaTeXe{}'s \tn{@gobble}, \tn{@gobbbletwo},
    \emph{etc.}
  \end{texnote}
\end{function}

\begin{function}[EXP, added = 2018-06-18]{\use:e}
  \begin{syntax}
    \cs{use:e} \Arg{expandable tokens}
  \end{syntax}
  Fully expands the \meta{token list} in an \texttt{x}-type manner,
  \emph{but} the function remains fully expandable, and parameter
  character (usually~|#|) need not be doubled.
  \begin{texnote}
    \cs{use:e} is a wrapper around the primitive \tn{expanded}
    where it is available:
    it requires two expansions to complete its action.
    When \tn{expanded} is not available this function is very slow.
  \end{texnote}
\end{function}

\begin{function}[updated = 2011-12-31]{\use:x}
  \begin{syntax}
    \cs{use:x} \Arg{expandable tokens}
  \end{syntax}
  Fully expands the \meta{expandable tokens} and inserts the
  result into the input stream at the current location.
  Any hash characters (|#|) in the argument must be doubled.
\end{function}

\subsection{Selecting tokens from delimited arguments}

A different kind of function for selecting tokens from the token
stream are those that use delimited arguments.

\begin{function}[EXP]
  {
    \use_none_delimit_by_q_nil:w,
    \use_none_delimit_by_q_stop:w,
    \use_none_delimit_by_q_recursion_stop:w
  }
  \begin{syntax}
   \cs{use_none_delimit_by_q_nil:w} \meta{balanced text} \cs{q_nil}
   \cs{use_none_delimit_by_q_stop:w} \meta{balanced text} \cs{q_stop}
   \cs{use_none_delimit_by_q_recursion_stop:w} \meta{balanced text} \cs{q_recursion_stop}
  \end{syntax}
  Absorb the \meta{balanced text} form the input stream delimited by
  the marker given in the function name, leaving nothing in the
  input stream.
\end{function}

\begin{function}[EXP]
  {
    \use_i_delimit_by_q_nil:nw,
    \use_i_delimit_by_q_stop:nw,
    \use_i_delimit_by_q_recursion_stop:nw
  }
  \begin{syntax}
   \cs{use_i_delimit_by_q_nil:nw} \Arg{inserted tokens} \meta{balanced text} \cs{q_nil}
   \cs{use_i_delimit_by_q_stop:nw} \Arg{inserted tokens} \meta{balanced text} \cs{q_stop}
   \cs{use_i_delimit_by_q_recursion_stop:nw} \Arg{inserted tokens} \meta{balanced text} \cs{q_recursion_stop}
  \end{syntax}
  Absorb the \meta{balanced text} form the input stream delimited by
  the marker given in the function name, leaving \meta{inserted tokens}
  in the input stream for further processing.
\end{function}

\section{Predicates and conditionals}

\LaTeX3 has three concepts for conditional flow processing:
\begin{description}
  \item[Branching conditionals]
    Functions that carry out a test and then execute, depending on its
    result, either the code supplied as the \meta{true code} or the
    \meta{false code}.
    These arguments are denoted with |T| and |F|, respectively. An
    example would be
    \begin{quote}
     |\cs_if_free:cTF {abc}| \Arg{true code} \Arg{false code}
    \end{quote}
    a function that turns the first argument into a control sequence
    (since it's marked as |c|) then checks whether this control sequence
    is still free and then depending on the result carries out the code in
    the second argument (true case) or in the third argument (false
    case).

    These type of functions are known as \enquote{conditionals};
    whenever a |TF| function is defined it is usually accompanied by
    |T| and |F| functions as well. These are provided for convenience when
    the branch only needs to go a single way. Package writers are free to
    choose which types to define but the kernel definitions always
    provide all three versions.

    Important to note is that these branching conditionals with \meta{true
    code} and/or \meta{false code} are always defined in a way that the
    code of the chosen alternative can operate on following tokens in
    the input stream.

    These conditional functions may or may not be fully expandable, but if
    they are expandable they are accompanied by a \enquote{predicate}
    for the same test as described below.

  \item[Predicates]
    \enquote{Predicates} are functions that return a special type of
    boolean value which can be tested by the boolean expression parser.
    All functions of this type
    are expandable and have names that end with |_p| in the
    description part.  For example,
    \begin{quote}
      \cs{cs_if_free_p:N}
    \end{quote}
    would be a predicate function for the same type of test as the
    conditional described above. It would return \enquote{true} if its
    argument (a single token denoted by |N|) is still free for definition.
    It would be used in constructions like
    \begin{quote}
      |\bool_if:nTF {| \\
      \verb"  \cs_if_free_p:N \l_tmpz_tl || \cs_if_free_p:N \g_tmpz_tl " \\
      |}|
      \Arg{true code} \Arg{false code}
    \end{quote}

    For each predicate defined, a \enquote{branching conditional}
    also exists that behaves like a conditional described above.

  \item[Primitive conditionals]
     There is a third variety of conditional, which is the original
     concept used in plain \TeX{} and \LaTeXe{}. Their use is discouraged
     in \pkg{expl3} (although still used in low-level definitions)
     because they are more fragile and in many cases require more
     expansion control (hence more code) than the two types of
     conditionals described above.
\end{description}

\begin{variable}{\c_true_bool, \c_false_bool}
  Constants that represent |true| and |false|, respectively. Used to
  implement predicates.
\end{variable}

\subsection{Tests on control sequences}

\begin{function}[EXP,pTF]{\cs_if_eq:NN}
  \begin{syntax}
    \cs{cs_if_eq_p:NN} \meta{cs_1} \meta{cs_2}
    \cs{cs_if_eq:NNTF} \meta{cs_1} \meta{cs_2} \Arg{true code} \Arg{false code}
  \end{syntax}
  Compares the definition of two \meta{control sequences} and
  is logically \texttt{true} if they are the same, \emph{i.e.}~if they have exactly
  the same definition when examined with \cs{cs_show:N}.
\end{function}

\begin{function}[EXP,pTF]{\cs_if_exist:N, \cs_if_exist:c}
  \begin{syntax}
    \cs{cs_if_exist_p:N} \meta{control sequence}
    \cs{cs_if_exist:NTF} \meta{control sequence} \Arg{true code} \Arg{false code}
  \end{syntax}
  Tests whether the \meta{control sequence} is currently defined
  (whether as a function or another control sequence type). Any
  definition of \meta{control sequence} other than \tn{relax}
  evaluates as \texttt{true}.
\end{function}

\begin{function}[EXP,pTF]{\cs_if_free:N, \cs_if_free:c}
  \begin{syntax}
    \cs{cs_if_free_p:N} \meta{control sequence}
    \cs{cs_if_free:NTF} \meta{control sequence} \Arg{true code} \Arg{false code}
  \end{syntax}
  Tests whether the \meta{control sequence} is currently free to
  be defined. This test is \texttt{false} if the
  \meta{control sequence} currently exists (as defined by
  \cs{cs_if_exist:N}).
\end{function}

\subsection{Primitive conditionals}

The \eTeX{} engine itself provides many different conditionals. Some
expand whatever comes after them and others don't. Hence the names
for these underlying functions often contains a |:w| part but
higher level functions are often available. See for instance
\cs{int_compare_p:nNn} which is a wrapper for \cs{if_int_compare:w}.

Certain conditionals deal with specific data types like boxes and
fonts and are described there. The ones described below are either
the universal conditionals or deal with control sequences. We
prefix primitive conditionals with |\if_|.

\begin{function}[EXP]
  {\if_true:, \if_false:, \else:, \fi:, \reverse_if:N}
  \begin{syntax}
    "\if_true:" <true code> "\else:" <false code> "\fi:" \\
    "\if_false:" <true code> "\else:" <false code> "\fi:" \\
    "\reverse_if:N" <primitive conditional>
  \end{syntax}
  "\if_true:" always executes <true code>, while "\if_false:" always
  executes <false code>. "\reverse_if:N" reverses any two-way primitive
  conditional. "\else:" and "\fi:" delimit the branches of the
  conditional. The function "\or:" is documented in \pkg{l3int} and
  used in case switches.
  \begin{texnote}
    These are equivalent to their corresponding \TeX\ primitive
    conditionals; \cs{reverse_if:N} is \eTeX's \tn{unless}.
  \end{texnote}
\end{function}

\begin{function}[EXP]{\if_meaning:w}
  \begin{syntax}
    "\if_meaning:w" <arg_1> <arg_2> <true code> "\else:" <false code> "\fi:"
  \end{syntax}
  "\if_meaning:w" executes <true code> when <arg_1> and <arg_2> are the same,
  otherwise it executes <false code>.
  <arg_1> and <arg_2> could be functions, variables, tokens; in all cases the
  \emph{unexpanded} definitions are compared.
  \begin{texnote}
    This is \TeX{}'s \tn{ifx}.
  \end{texnote}
\end{function}

\begin{function}[EXP]{\if:w, \if_charcode:w, \if_catcode:w}
  \begin{syntax}
    "\if:w" <token_1> <token_2> <true code> "\else:" <false code> "\fi:" \\
    "\if_catcode:w" <token_1> <token_2> <true code> "\else:" <false code> "\fi:"
  \end{syntax}
  These conditionals expand any following tokens until two
  unexpandable tokens are left. If you wish to prevent this expansion,
  prefix the token in question with "\exp_not:N". "\if_catcode:w"
  tests if the category codes of the two tokens are the same whereas
  "\if:w" tests if the character codes are
  identical. "\if_charcode:w" is an alternative name for "\if:w".
\end{function}

\begin{function}[EXP]{\if_cs_exist:N, \if_cs_exist:w}
  \begin{syntax}
    "\if_cs_exist:N" <cs> <true code> "\else:" <false code> "\fi:" \\
    "\if_cs_exist:w" <tokens> "\cs_end:" <true code> "\else:" <false code> "\fi:"
  \end{syntax}
  Check if <cs> appears in the hash table or if the control sequence
  that can be formed from <tokens> appears in the hash table. The
  latter function does not turn the control sequence in question into
  "\scan_stop:"! This can be useful when dealing with control
  sequences which cannot be entered as a single token.
\end{function}

\begin{function}[EXP]
  {
    \if_mode_horizontal:, \if_mode_vertical:,
    \if_mode_math:, \if_mode_inner:
  }
  \begin{syntax}
    "\if_mode_horizontal:" <true code> "\else:" <false code> "\fi:"
  \end{syntax}
  Execute <true code> if currently in horizontal mode, otherwise
  execute <false code>. Similar for the other functions.
\end{function}

\section{Starting a paragraph}

\begin{function}[added = 2017-07-04]{\mode_leave_vertical:}
  \begin{syntax}
    \cs{mode_leave_vertical:}
  \end{syntax}
  Ensures that \TeX{} is not in vertical (inter-paragraph) mode. In
  horizontal or math mode this command has no effect, in vertical mode it
  switches to horizontal mode, and inserts a box of width
  \tn{parindent}, followed by the \tn{everypar} token list.
  \begin{texnote}
    This results in the contents of the \tn{everypar} token register being
    inserted, after \cs{mode_leave_vertical:} is complete. Notice that in
    contrast to the \LaTeXe{} \tn{leavevmode} approach, no box is used
    by the method implemented here.
  \end{texnote}
\end{function}

\subsection{Debugging support}

\begin{function}[added = 2017-07-16, updated = 2017-08-02]{\debug_on:n, \debug_off:n}
  \begin{syntax}
    \cs{debug_on:n} |{| \meta{comma-separated list} |}|
    \cs{debug_off:n} |{| \meta{comma-separated list} |}|
  \end{syntax}
  Turn on and off within a group various debugging code, some of which
  is also available as \pkg{expl3} load-time options.  The items that
  can be used in the \meta{list} are
  \begin{itemize}
    \item \texttt{check-declarations} that checks all \pkg{expl3}
      variables used were previously declared and that local/global
      variables (based on their name or on their first assignment) are
      only locally/globally assigned;
    \item \texttt{check-expressions} that checks integer, dimension,
      skip, and muskip expressions are not terminated prematurely;
    \item \texttt{deprecation} that makes soon-to-be-deprecated commands produce errors;
    \item \texttt{log-functions} that logs function definitions;
    \item \texttt{all} that does all of the above.
  \end{itemize}
  Providing these as switches rather than options allows testing code
  even if it relies on other packages: load all other packages, call
  \cs{debug_on:n}, and load the code that one is interested in
  testing.  These functions can only be used in \LaTeXe{} package mode
  loaded with \texttt{enable-debug} or another option implying it.
\end{function}

\begin{function}[added = 2017-11-28]{\debug_suspend:, \debug_resume:}
  \begin{syntax}
    \cs{debug_suspend:} \ldots{} \cs{debug_resume:}
  \end{syntax}
  Suppress (locally) errors and logging from \texttt{debug} commands,
  except for the \texttt{deprecation} errors or warnings.  These pairs
  of commands can be nested.  This can be used around pieces of code
  that are known to fail checks, if such failures should be ignored.
  See for instance \pkg{l3coffins}.
\end{function}

\end{documentation}

\end{document}
