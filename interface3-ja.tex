% +++
% sequence = ["latex", "dvipdf"]
% latex = "uplatex"
% clean_files = [
%   "%B.aux", "%B.dvi", "%B.glo", "%B.hd", "%B.idx", "%B.ind",
%   "%B.ilg", "%B.log", "%B.out", "%B.synctex.gz",
% ]
% +++
\documentclass[dvipdfmx,kernel]{wtpl3doc}
\usepackage{docmute}
%\listfiles

\begin{document}

\title{The \LaTeX3 Interfaces}
\author{%
 The \LaTeX3 Project\thanks
   {%
     E-mail:
       \href{mailto:latex-team@latex-project.org}
         {latex-team@latex-project.org}%
   }%
}
\date{Released 2020-01-22}

\pagenumbering{roman}
\maketitle

\begin{abstract}

\setlength\parindent{0pt}
\setlength\parskip{\baselineskip}

\noindent
This is the reference documentation for the \pkg{expl3}
programming environment. The \pkg{expl3} modules set up an experimental
naming scheme for \LaTeX{} commands, which allow the \LaTeX{} programmer
to systematically name functions and variables, and specify the argument
types of functions.

The \TeX{} and \eTeX{} primitives are all given a new name according to
these conventions. However, in the main direct use of the primitives is
not required or encouraged: the \pkg{expl3} modules define an
independent low-level \LaTeX3 programming language.

At present, the \pkg{expl3} modules are designed to be loaded on top of
\LaTeXe{}. In time, a \LaTeX3 format will be produced based on this code.
This allows the code to be used in \LaTeXe{} packages \emph{now} while a
stand-alone \LaTeX3 is developed.

\begin{bfseries}
  While \pkg{expl3} is still experimental, the bundle is now regarded as
  broadly stable. The syntax conventions and functions provided are now
  ready for wider use. There may still be changes to some functions, but
  these will be minor when compared to the scope of \pkg{expl3}.

  New modules will be added to the distributed version of \pkg{expl3} as
  they reach maturity.
\end{bfseries}

\end{abstract}

\clearpage

{%
  \def\\{:}% fix "newlines" in the ToC
  \tableofcontents
}

\clearpage
\pagenumbering{arabic}

\makeatletter
\def\partname{Part}
\def\maketitle{\clearpage\part{\@title}}
\let\thanks\@gobble
\let\DelayPrintIndex\PrintIndex
\let\PrintIndex\@empty
\providecommand*{\hexnum}[1]{\text{\texttt{\char`\"}#1}}
\makeatother

% +++
% sequence = ["latex", "dvipdf"]
% latex = "uplatex"
% clean_files = [
%   "%B.aux", "%B.dvi", "%B.glo", "%B.hd", "%B.idx", "%B.ind",
%   "%B.ilg", "%B.log", "%B.out", "%B.synctex.gz",
% ]
% +++
\documentclass[uplatex,dvipdfmx,full,kernel]{wtpl3doc}

\usepackage{enumitem}

\begin{document}

%\title{Introduction to \pkg{expl3} and this document}
\title{\pkg{expl3}および本ドキュメントのイントロダクション}
%\author{%
% The \LaTeX3 Project\thanks
%   {%
%     E-mail:
%       \href{mailto:latex-team@latex-project.org}
%         {latex-team@latex-project.org}%
%   }%
%}
\author{%
 \LaTeX3プロジェクト\thanks
   {%
     E-mail:
       \href{mailto:latex-team@latex-project.org}
         {latex-team@latex-project.org}%
   }%
}
%\date{Released 2020-01-22}
\date{バージョン 2020-01-22}

\maketitle

%This document is intended to act as a comprehensive reference manual
%for the \pkg{expl3} language. A general guide to the \LaTeX3
%programming language is found in \href{expl3.pdf}{expl3.pdf}.
このドキュメントは\pkg{expl3}言語の網羅的なリファレンスマニュアルとして
作成されたものです.\LaTeX3プログラミング言語の一般的な概説については
\href{expl3.pdf}{expl3.pdf}を参照してください.

%\section{Naming functions and variables}
\section{関数と変数の名前}

%\LaTeX3 does not use \texttt{@} as a \enquote{letter} for defining
%internal macros.  Instead, the symbols |_| and \texttt{:}
%are used in internal macro names to provide structure. The name of
%each \emph{function} is divided into logical units using \texttt{_},
%while \texttt{:} separates the \emph{name} of the function from the
%\emph{argument specifier} (\enquote{arg-spec}). This describes the arguments
%expected by the function. In most cases, each argument is represented
%by a single letter. The complete list of arg-spec letters for a function
%is referred to as the \emph{signature} of the function.
\LaTeX3では\code{@}を内部マクロ定義のための\enquote{letter}として使用
しません.代わりに|_|と\code{:}が内部マクロに構造を持たせる目的で使用
されます.各関数の名前は|_|によって論理的な単位ごとに区切られ,さらに
\code{:}によって関数の\emph{名前}と\emph{引数指定子}
(\enquote{arg-spec}) に分けられます.引数指定子は,その関数が取る引数
の形を表すものです.ほとんどの場合,1つの引数は1文字の引数指定子に
よって表現されます.ある関数が取る引数の数だけ引数指定子を並べた文字列
は,その関数の\emph{サイン}と見なすことができます.

%Each function name starts with the \emph{module} to which it belongs.
%Thus apart from a small number of very basic functions, all \pkg{expl3}
%function names contain at least one underscore to divide the module
%name from the descriptive name of the function. For example, all
%functions concerned with comma lists are in module \texttt{clist} and
%begin |\clist_|.
各関数の名前は,その関数が属する\emph{モジュール}名から始まります.
したがって,少数のごく基本的な関数を除き,すべての\pkg{expl3}関数名
にはモジュール名の部分を関数の説明的な名前と分離するためのアンダー
スコアが少なくとも1つ含まれています.例えば,カンマ区切りリストを扱う
すべての関数は\code{clist}モジュールに含まれていて,それらの名前は必ず
|\clist_|で始まります.

%Every function must include an argument specifier. For functions which
%take no arguments, this will be blank and the function name will end
%\texttt{:}. Most functions take one or more arguments, and use the
%following argument specifiers:
すべての関数名には引数指定子が含まれていなければなりません.引数を
取らない関数については,引数指定子の部分は空文字列となり,その関数名は
\code{:}で終わることになります.多くの関数は1つ以上の引数を取り,以下の
引数指定子を使用することになります:
%
\begin{description}[style=multiline, leftmargin=3zw]
%  \item[\texttt{N} and \texttt{n}] These mean \emph{no manipulation},
%    of a single token for \texttt{N} and of a set of tokens given in
%    braces for \texttt{n}. Both pass the argument through exactly as
%    given. Usually, if you use a single token for an \texttt{n} argument,
%    all will be well.
  \item[\code{N}/\code{n}] これらは\emph{操作なし} (no manipulation)
    を意味します.\code{N}は単一トークン,\code{n}はブレースで囲われた
    トークン列です.いずれの引数も関数にそのまま渡されることになります.
    通常,\code{n}タイプの引数に単一トークンを与えたとしても,特に問題
    は起こりません.
%
%  \item[\texttt{c}] This means \emph{csname}, and indicates that the
%    argument will be turned into a csname before being used. So
%    |\foo:c| |{ArgumentOne}| will act in the same way as |\foo:N|
%    |\ArgumentOne|.
  \item[\code{c}] これは\emph{制御綴名} (csname) の意味で,このタイプの
    引数は使用される前に制御綴名に変換されることを示しています.つまり
    |\foo:c {ArgumentOne}|は|\foo:N \ArgumentOne|とまったく同じ挙動を
    することになります.
%
%  \item[\texttt{V} and \texttt{v}] These mean \emph{value
%    of variable}. The \texttt{V} and \texttt{v} specifiers are used to
%    get the content of a variable without needing to worry about the
%    underlying \TeX{} structure containing the data. A \texttt{V}
%    argument will be a single token (similar to \texttt{N}), for example
%    |\foo:V| |\MyVariable|; on the other hand, using \texttt{v} a
%    csname is constructed first, and then the value is recovered, for
%    example |\foo:v| |{MyVariable}|.
  \item[\code{V}/\code{v}] これらは\emph{変数の値} (value of variable)
    を意味しています.この2つの指定子は,ある変数がもつ値を,その実装に
    用いられている\TeX レベルの構造を考慮することなく取得するためのもの
    です.\code{V}タイプの引数は(\code{N}と同様の)単一トークンです.
    |\foo:V \MyVariable|のように使用します.一方で\code{v}を使用した
    場合は,まず制御綴に変換され,その上で変数の値が取り出されます.
    具体的には|\foo:v {MyVariable}|のように使用します.
%
%  \item[\texttt{o}] This means \emph{expansion once}. In general, the
%    \texttt{V} and \texttt{v} specifiers are favoured over \texttt{o}
%    for recovering stored information. However, \texttt{o} is useful
%    for correctly processing information with delimited arguments.
  \item[\code{o}] これは\emph{一回展開} (expand once) を意味します.
    一般には変数に格納されている情報を取り出す際には\code{V}/\code{v}を
    用いる方が好ましいです.しかし\code{o}はデリミタ区切りの引数を正しく
    処理する際に有用です.
%
%  \item[\texttt{x}] The \texttt{x} specifier stands for \emph{exhaustive
%    expansion}: every token in the argument is fully expanded until only
%    unexpandable ones remain. The \TeX{} \tn{edef} primitive carries out
%    this type of expansion. Functions which feature an \texttt{x}-type
%    argument are \emph{not} expandable.
  \item[\code{x}] これは\emph{完全展開} (exhaustive expansion) を意味して
    います.このタイプの引数に含まれるすべてのトークンは展開不能なもののみ
    が残る状態まで徹底的に展開されます.この動作は\TeX の\tn{edef}プリミ
    ティブによって実現されています.\code{x}タイプの引数をとる関数は展開
    可能では\emph{ありません}.
%
%  \item[\texttt{e}] The \texttt{e} specifier is in many respects
%    identical to \texttt{x}, but with a very different implementation.
%    Functions which feature an \texttt{e}-type argument may be
%    expandable.  The drawback is that \texttt{e} is extremely slow
%    (often more than $200$ times slower) in older engines, more
%    precisely in non-\LuaTeX{} engines older than 2019.
  \item[\code{e}] これはさまざまな点で\code{x}と同様ですが,それぞれの
    実装は大きく異なっています.このタイプの引数をもつ関数は完全展開
    可能たり得ます.タイプ\code{e}の欠点は古い\TeX 処理系(具体的には
    2019年以前の\LuaTeX 以外の処理系)では処理が極めて低速(200倍以上
    遅いことが多いです)であることです.
%
%  \item[\texttt{f}] The \texttt{f} specifier stands for \emph{full
%    expansion}, and in contrast to \texttt{x} stops at the first
%    non-expandable token (reading the argument from left to right) without
%    trying to expand it. If this token is a \meta{space token}, it is gobbled,
%    and thus won't be part of the resulting argument. For example, when
%    setting a token list variable (a macro used for storage), the sequence
  \item[\code{f}] これは\emph{先頭完全展開} (full expansion) を意味して
    います.\code{x}の場合と異なり,(引数を左から右に読んでいき)最初の
    展開不能トークンが現れたところで展開が停止します.もし最初の展開不能
    トークンが\meta{space token}であった場合は,そのトークンは取り除かれ
    最終的な引数には残りません.例えば,トークンリスト変数(値の保存に
    用いられるマクロ)を設定する
%
    \begin{verbatim}
      \tl_set:Nn \l_mya_tl { A }
      \tl_set:Nn \l_myb_tl { B }
      \tl_set:Nf \l_mya_tl { \l_mya_tl \l_myb_tl }
    \end{verbatim}
%
%    will leave |\l_mya_tl| with the content |A\l_myb_tl|, as |A| cannot
%    be expanded and so terminates expansion before |\l_myb_tl| is considered.
    というコードは|\l_mya_tl|の値を|A\l_myb_tl|に設定します.|A|は展開不能
    トークンなので,|\l_myb_tl|を展開する前に展開処理が終了するためです.
%
%  \item[\texttt{T} and \texttt{F}] For logic tests, there are the branch
%    specifiers \texttt{T} (\emph{true}) and \texttt{F} (\emph{false}).
%    Both specifiers treat the input in the same way as \texttt{n} (no
%    change), but make the logic much easier to see.
  \item[\code{T}/\code{F}] 論理テストのために,分岐指定子\code{T} (true)
    と\code{F} (false) が用意されています.これらの指定子はその入力を
    \code{n}と同様に扱います(つまり,何も変更しません)が,分岐処理を
    かなり見やすくします.
%
%  \item[\texttt{p}] The letter \texttt{p} indicates \TeX{}
%    \emph{parameters}. Normally this will be used for delimited
%    functions as \pkg{expl3} provides better methods for creating simple
%    sequential arguments.
  これは\TeX のパラメータ (parameters) を意味しています.通常この指定子は
  デリミタ付き引数を取る関数定義に使用されますが,\pkg{expl3}にはシンプル
  な関数に対してはより洗練された方法を提供しています.
%
%  \item[\texttt{w}] Finally, there is the \texttt{w} specifier for
%    \emph{weird} arguments. This covers everything else, but mainly
%    applies to delimited values (where the argument must be terminated
%    by some specified string).
  \item[\code{w}] これは\emph{奇妙な} (weird) 引数のための指定子です.
    これは上記の分類に当てはまらないすべての引数を表現するためのものです
    が,主にデリミタ区切りにされる値(そうした引数は必ず特定の文字列で
    終端が示される必要があります)を扱う際に採用されます.
%
%  \item[\texttt{D}] The \texttt{D} specifier means \emph{do not use}.
%    All of the \TeX{} primitives are initially \cs{let} to a \texttt{D}
%    name, and some are then given a second name.  Only the kernel
%    team should use anything with a \texttt{D} specifier!
  \item[\code{D}] これは\emph{使用不可} (do not use) を意味します.
    すべての\TeX プリミティブは最初に引数指定子\code{D}をもつ名前に
    \cs{let}代入され,一部にはさらに別の名称が与えられます.\LaTeX3の
    カーネル開発者のみが\code{D}指定子をもつ関数を使用すべきです.
\end{description}
%
%Notice that the argument specifier describes how the argument is
%processed prior to being passed to the underlying function. For example,
%|\foo:c| will take its argument, convert it to a control sequence and
%pass it to |\foo:N|.
引数指定子は,それぞれの引数がその関数本体に渡される前にどう処理されるか
を示すものであるということに注意してください.例えば|\foo:c|は最初の引数
をとり,それを制御綴に変換したものを|\foo:N|に渡します.

%Variables are named in a similar manner to functions, but begin with
%a single letter to define the type of variable:
変数の命名規則も関数のそれと似ていますが,変数の種類を表す1文字から始まり
ます:
%
\begin{description}[style=multiline, leftmargin=3zw]
%  \item[\texttt{c}] Constant: global parameters whose value should not
%    be changed.
%  \item[\texttt{g}] Parameters whose value should only be set globally.
%  \item[\texttt{l}] Parameters whose value should only be set locally.
  \item[\code{c}] 定数.グローバルで,値は変更されるべきでありません.
  \item[\code{g}] グローバル変数.その値は必ずグローバルに設定されるべき
    です.
  \item[\code{l}] ローカル変数.その値は必ずローカルに設定されるべきです.
\end{description}
%
%Each variable name is then build up in a similar way to that of a
%function, typically starting with the module\footnote{The module names are
%  not used in case of generic scratch registers defined in the data
%  type modules, e.g., the
%  \texttt{int} module contains some scratch variables called \cs{l_tmpa_int},
%  \cs{l_tmpb_int}, and so on. In such a case adding the module name up front
%  to denote the module
%  and in the back to indicate the type, as in
%  \texttt{\string\l_int_tmpa_int} would be very unreadable.}  name
%and then a descriptive part.
%Variables end with a short identifier to show the variable type:
上記の種別の後続部は,各変数の名前は関数と似た構成をもちます.典型的には
まずモジュール名\footnote{モジュール名は各データ型モジュールにおいて汎用
目的で確保されているレジスタでは省略されています.例えば\code{int}モジュール
には\cs{l_tmpa_int}, \cs{l_tmpb_int}という汎用の一時変数が含まれています.
このような変数については,先頭にモジュール名・末尾にデータ型をおいてしまうと
\code{\string\l_int_tmpa_int}のように極めて不格好な名前になってしまいます}が
おかれ,続いて説明的な名前がおかれます.変数名の末尾にはデータ型を表す短い
識別子がおかれます:
%
\begin{description}[style=multiline, leftmargin=5zw]
%  \item[\texttt{clist}] Comma separated list.
%  \item[\texttt{dim}] \enquote{Rigid} lengths.
%  \item[\texttt{fp}] Floating-point values;
%  \item[\texttt{int}] Integer-valued count register.
%  \item[\texttt{muskip}] \enquote{Rubber} lengths for use in
%    mathematics.
%  \item[\texttt{seq}] \enquote{Sequence}: a data-type used to implement
%    lists (with access at both ends) and stacks.
%  \item[\texttt{skip}] \enquote{Rubber} lengths.
%  \item[\texttt{str}] String variables: contain character data.
%  \item[\texttt{tl}] Token list variables: placeholder for a token list.
  \item[\code{clist}] カンマ区切りリスト
  \item[\code{dim}] 寸法 (\enquote{rigid} length) 
  \item[\code{fp}] 浮動小数点数
  \item[\code{int}] 整数カウンタレジスタ
  \item[\code{muskip}] 数式用スキップ (\enquote{rubber} length)
  \item[\code{seq}] \emph{シークエンス}:両端からアクセス可能なリストと
    スタックを実装するのに用いられるデータ構造
  \item[\code{skip}] スキップ (\enquote{rubber} length)
  \item[\code{str}] 文字列変数:文字データを含むもの
  \item[\code{tl}] トークンリスト変数:トークンリストを格納するもの
\end{description}
%
%Applying \texttt{V}-type or \texttt{v}-type expansion to variables of
%one of the above types is supported, while it is not supported for the
%following variable types:
上記のデータ型の変数については\code{V}タイプおよび\code{v}タイプの展開が
サポートされていますが,以下のデータ型についてはサポートされていません:
%
\begin{description}[style=multiline, leftmargin=5zw]
%  \item[\texttt{bool}] Either true or false.
%  \item[\texttt{box}] Box register.
%  \item[\texttt{coffin}] A \enquote{box with handles} --- a higher-level
%    data type for carrying out \texttt{box} alignment operations.
%  \item[\texttt{flag}] Integer that can be incremented expandably.
%  \item[\texttt{fparray}] Fixed-size array of floating point values.
%  \item[\texttt{intarray}] Fixed-size array of integers.
%  \item[\texttt{ior}/\texttt{iow}] An input or output stream, for
%    reading from or writing to, respectively.
%  \item[\texttt{prop}] Property list: analogue of dictionary or
%    associative arrays in other languages.
%  \item[\texttt{regex}] Regular expression.
\item[\code{bool}] 真 (true) または偽 (false)
\item[\code{box}] ボックスレジスタ
\item[\code{coffin}] 高機能ボックス:ボックス配列操作のための高級なデータ
  構造
\item[\code{flag}] 展開限定文脈でインクリメント可能な整数
\item[\code{fparray}] 浮動小数点数用の静的配列
\item[\code{intarray}] 整数用の静的配列
\item[\code{ior}/\code{iow}] 入出力ストリーム
\item[\code{prop}] プロパティリスト:他言語の辞書や連想配列に相当
\item[\code{regex}] 正規表現
\end{description}

%\subsection{Terminological inexactitude}
\subsection{言語的不正確 (terminological inexactitude)}

%A word of warning. In this document, and others referring to the \pkg{expl3}
%programming modules, we often refer to \enquote{variables} and
%\enquote{functions} as if they
%were actual constructs from a real programming language.  In truth, \TeX{}
%is a macro processor, and functions are simply macros that may or may not take
%arguments and expand to their replacement text.  Many of the common variables
%are \emph{also} macros, and if placed into the input stream will simply expand
%to their definition as well~--- a \enquote{function} with no arguments and a
%\enquote{token list variable} are almost the same.\footnote{\TeX{}nically,
%functions with no arguments are \tn{long} while token list variables are not.}
%On the other
%hand, some \enquote{variables} are actually registers that must be
%initialised and their values set and retrieved with specific functions.
これは警告です.このドキュメント,およびその他の\pkg{expl3}プログラミング
モジュールについて述べる文書において,しばしば「変数」や「関数」という用語は
それらがあたかも真のプログラミング言語の構成要素であるかのように使用されます.
しかし実際には,\TeX はマクロ処理系であり,関数は時に引数をとり,また時に
引数をとらずにその置換テキストへと展開されます.多くの一般的なマクロも
\emph{やはり}マクロであり,入力ストリームに置かれた場合はその定義に従って
関数と同様に展開されることになります.つまり引数のない「関数」と「トークン
リスト変数」はほぼ同じものです\footnote{{\TeX}nicalには引数のない関数は
\tn{long}有効なマクロなのに対し,トークンリスト変数はそうではないという
違いはあります.}.一方で,いくつかの変数は一部の「変数」は実体がレジスタ
で,使用前には初期化が,その値の設定や取得には専用の関数が必要です.

%The conventions of the \pkg{expl3} code are designed to clearly separate the
%ideas of \enquote{macros that contain data} and
%\enquote{macros that contain code}, and a
%consistent wrapper is applied to all forms of \enquote{data} whether they be
%macros or
%actually registers.  This means that sometimes we will use phrases like
%\enquote{the function returns a value}, when actually we just mean
%\enquote{the macro expands to something}. Similarly, the term
%\enquote{execute} might be used in place of \enquote{expand}
%or it might refer to the more specific case of
%\enquote{processing in \TeX's stomach}
%(if you are familiar with the \TeX{}book parlance).
\pkg{expl3}コードの習慣は「データを格納するためのマクロ」と「コードを含む
マクロ」を明確に区別し,あらゆる「データ」について,その実体がマクロであれ
レジスタであれ一貫したラッパををもつように設計されています.そのため時として
実際には単に「マクロが何かに展開される」ことを「関数が値を返す」というように
表現する場合があります.同様に「実行する」という表現は単に「展開」を意味して
いるか,あるいは特定のケースでは「\TeX のお腹で処理が行なわれる」(これは
『\TeX ブック』の用語に詳しい人向けの表現です)ということを意味しています.

%If in doubt, please ask; chances are we've been hasty in writing certain
%definitions and need to be told to tighten up our terminology.
もしこの方針を疑うならば,お知らせください.もしかすると,私たちはより厳密な
定義を書く下す余裕がないほどせっかちだったかもしれません.その場合,私たちの
用語を整理する必要があると指摘される必要があります.
%% !! この訳は自信がない !!

\section{Documentation conventions}

This document is typeset with the experimental \pkg{l3doc} class;
several conventions are used to help describe the features of the code.
A number of conventions are used here to make the documentation clearer.

Each group of related functions is given in a box. For a function with
a \enquote{user} name, this might read:
\begin{function}[label = ]{\ExplSyntaxOn, \ExplSyntaxOff}
  \begin{syntax}
    |\ExplSyntaxOn| \dots{} |\ExplSyntaxOff|
  \end{syntax}
  The textual description of how the function works would appear here. The
  syntax of the function is shown in mono-spaced text to the right of
  the box. In this example, the function takes no arguments and so the
  name of the function is simply reprinted.
\end{function}

For programming functions, which use \texttt{_} and \texttt{:} in their name
there are a few additional conventions: If two related functions are given
with identical names but different argument specifiers, these are termed
\emph{variants} of each other, and the latter functions are printed in grey to
show this more clearly. They will carry out the same function but will take
different types of argument:
\begin{function}[label = ]{\seq_new:N, \seq_new:c}
  \begin{syntax}
    |\seq_new:N| \meta{sequence}
  \end{syntax}
  When a number of variants are described, the arguments are usually
  illustrated only for the base function. Here, \meta{sequence} indicates
  that |\seq_new:N| expects the name of a sequence. From the argument
  specifier, |\seq_new:c| also expects a sequence name, but as a
  name rather than as a control sequence. Each argument given in the
  illustration should be described in the following text.
\end{function}

\paragraph{Fully expandable functions}
\hypertarget{expstar}{Some functions are fully expandable},
which allows them to be used within
an \texttt{x}-type or \texttt{e}-type argument (in plain \TeX{} terms, inside an \tn{edef} or \tn{expanded}),
as well as within an \texttt{f}-type argument.
These fully expandable functions are indicated in the documentation by
a star:
\begin{function}[EXP, label = ]{\cs_to_str:N}
  \begin{syntax}
    |\cs_to_str:N| \meta{cs}
  \end{syntax}
  As with other functions, some text should follow which explains how
  the function works. Usually, only the star will indicate that the
  function is expandable. In this case, the function expects a \meta{cs},
  shorthand for a \meta{control sequence}.
\end{function}

\paragraph{Restricted expandable functions}
\hypertarget{rexpstar}{A few functions are fully expandable} but cannot be fully expanded within
an \texttt{f}-type argument. In this case a hollow star is used to indicate
this:
\begin{function}[rEXP, label = ]{\seq_map_function:NN}
  \begin{syntax}
    |\seq_map_function:NN| \meta{seq} \meta{function}
  \end{syntax}
\end{function}

\paragraph{Conditional functions}
\hypertarget{explTF}{Conditional (\texttt{if}) functions}
are normally defined in three variants, with
\texttt{T}, \texttt{F} and \texttt{TF} argument specifiers. This allows
them to be used for different \enquote{true}/\enquote{false} branches,
depending on
which outcome the conditional is being used to test. To indicate this
without repetition, this information is given in a shortened form:
\begin{function}[EXP,TF, label = ]{\sys_if_engine_xetex:}
  \begin{syntax}
    |\sys_if_engine_xetex:TF| \Arg{true code} \Arg{false code}
  \end{syntax}
  The underlining and italic of \texttt{TF} indicates that three functions
  are available:
  \begin{itemize}
    \item |\sys_if_engine_xetex:T|
    \item |\sys_if_engine_xetex:F|
    \item |\sys_if_engine_xetex:TF|
  \end{itemize}
  Usually, the illustration
  will use the \texttt{TF} variant, and so both \meta{true code}
  and \meta{false code} will be shown. The two variant forms \texttt{T} and
  \texttt{F} take only \meta{true code} and \meta{false code}, respectively.
  Here, the star also shows that this function is expandable.
  With some minor exceptions, \emph{all} conditional functions in the
  \pkg{expl3} modules should be defined in this way.
\end{function}

Variables, constants and so on are described in a similar manner:
\begin{variable}[label = ]{\l_tmpa_tl}
  A short piece of text will describe the variable: there is no
  syntax illustration in this case.
\end{variable}

In some cases, the function is similar to one in \LaTeXe{} or plain \TeX{}.
In these cases, the text will include an extra \enquote{\textbf{\TeX{}hackers
note}} section:
\begin{function}[EXP, label = ]{\token_to_str:N}
  \begin{syntax}
    |\token_to_str:N| \meta{token}
  \end{syntax}
  The normal description text.
  \begin{texnote}
    Detail for the experienced \TeX{} or \LaTeXe\ programmer. In this
    case, it would point out that this function is the \TeX{} primitive
    |\string|.
  \end{texnote}
\end{function}

\paragraph{Changes to behaviour}
When new functions are added to \pkg{expl3}, the date of first inclusion is
given in the documentation. Where the documented behaviour of a function
changes after it is first introduced, the date of the update will also be
given. This means that the programmer can be sure that any release of
\pkg{expl3} after the date given will contain the function of interest with
expected behaviour as described. Note that changes to code internals, including
bug fixes, are not recorded in this way \emph{unless} they impact on the
expected behaviour.

\section{Formal language conventions which apply generally}

As this is a formal reference guide for \LaTeX3 programming, the descriptions
of functions are intended to be reasonably \enquote{complete}. However, there
is also a need to avoid repetition. Formal ideas which apply to general
classes of function are therefore summarised here.

For tests which have a \texttt{TF} argument specification, the test if
evaluated to give a logically \texttt{TRUE} or \texttt{FALSE} result.
Depending on this result, either the \meta{true code} or the \meta{false code}
will be left in the input stream. In the case where the test is expandable,
and a predicate (|_p|) variant is available, the logical value determined by
the test is left in the input stream: this will typically be part of a larger
logical construct.

\section{\TeX{} concepts not supported by \LaTeX3{}}

The \TeX{} concept of an \enquote{\cs{outer}} macro is \emph{not supported}
at all by \LaTeX3{}. As such, the functions provided here may break when
used on top of \LaTeXe{} if \cs{outer} tokens are used in the arguments.

\end{document}

%\DocInput{l3bootstrap.dtx}
%\DocInput{l3names.dtx}
%\DocInput{l3basics.dtx}
%\DocInput{l3expan.dtx}
%\DocInput{l3tl.dtx}
%\DocInput{l3str.dtx}
%\DocInput{l3str-convert.dtx}
%\DocInput{l3quark.dtx}
%\DocInput{l3seq.dtx}
%\DocInput{l3int.dtx}
% +++
% sequence = ["latex", "dvipdf"]
% latex = "uplatex"
% clean_files = [
%   "%B.aux", "%B.dvi", "%B.glo", "%B.hd", "%B.idx", "%B.ind",
%   "%B.ilg", "%B.log", "%B.out", "%B.synctex.gz",
% ]
% +++
\documentclass[dvipdfmx,full,kernel]{wtpl3doc}

\begin{document}
\title{\pkg{l3int}パッケージ \\ 整数}
%
\author{%
 \LaTeX3プロジェクト\thanks
   {%
     E-mail:
       \href{mailto:latex-team@latex-project.org}
         {latex-team@latex-project.org}%
   }%
}
%
\date{バージョン 2020-01-22}
%
\maketitle
%
\begin{documentation}
%
%Calculation and comparison of integer values can be carried out
%using literal numbers, \texttt{int} registers, constants and
%integers stored in token list variables. The standard operators
%\texttt{+}, \texttt{-}, \texttt{/} and \texttt{*} and
%parentheses can be used within such expressions to carry
%arithmetic operations. This module carries out these functions
%on \emph{integer expressions} (\enquote{\texttt{intexpr}}).
整数値の計算や比較は数値リテラル,\code{int}レジスタ,定数,トークン
リストに格納された整数によって実現することができます.標準的な演算子
\code{+}, \code{-}, \code{/}, \code{*}およびカッコはそうした表現の
中で使用することができ,算術演算を可能にします.このモジュールは整数
表現 (integer expressions, \enquote{\code{intexpr}}) のための関数を
提供します.
%
%\section{Integer expressions}
\section{整数表現}
%
\begin{function}[EXP]{\int_eval:n}
  \begin{syntax}
    \cs{int_eval:n} \Arg{integer expression}
  \end{syntax}
%  Evaluates the \meta{integer expression} and leaves the result in the
%  input stream as an integer denotation: for positive results an
%  explicit sequence of decimal digits not starting with~\texttt{0},
%  for negative results \texttt{-}~followed by such a sequence, and
%  \texttt{0}~for zero.  The \meta{integer expression} should consist,
%  after expansion, of \texttt{+}, \texttt{-}, \texttt{*}, \texttt{/},
%  \texttt{(}, \texttt{)} and of course integer operands.  The result
%  is calculated by applying standard mathematical rules with the
%  following peculiarities:
  整数表現\meta{integer expression}を評価し,その結果を入力ストリームに
  明示的な整数値として残します.正の結果の場合は明示的な10進数,負の結果
  の場合は\code{-}に続く明示的な10進数,そして結果が0の場合は\code{0}が
  残ります.整数表現\meta{integer expression}は(展開された後)には整数の
  オペランドと\code{+}, \code{-}, \code{*}, \code{/}, \code{(}, \code{)}%
  を含めることができます.これらの演算結果は標準的な算術ルールに基づいて
  計算されますが,以下の点には注意が必要です:
%
  \begin{itemize}
%  \item \texttt{/} denotes division rounded to the closest integer with
%    ties rounded away from zero;
  \item \code{/}は除算を行いますが,その結果は最も近い整数に丸められます.
    ただし,差が等しい整数が2つある場合は0から遠い方に丸められます.
%
%  \item there is an error and the overall expression evaluates to zero
%    whenever the absolute value of any intermediate result exceeds
%    $2^{31}-1$, except in the case of scaling operations
%    $a$\texttt{*}$b$\texttt{/}$c$, for which $a$\texttt{*}$b$ may be
%    arbitrarily large;
  \item 演算には誤差があります.また計算の途中で$2^{31}-1$を超える結果が
    生じた場合,最終的な計算結果は0と評価されます.ただし$a$\code{*}$b$%
    \code{/}$c$のようなスケーリング操作については$a$\code{*}$b$の部分は
    任意の大きさになっても大丈夫です.
%
%  \item parentheses may not appear after unary \texttt{+} or
%    \texttt{-}, namely placing \texttt{+(} or \texttt{-(} at the start
%    of an expression or after \texttt{+}, \texttt{-}, \texttt{*},
%    \texttt{/} or~\texttt{(} leads to an error.
  \item カッコを単項演算子\code{+}, \code{-}の直後に置くことはできません.
    すなわち表現の冒頭や\code{+}, \code{-}, \code{*}, \code{/}, \code{(}%
    の直後に\code{+(}や\code{-(}を記述するとエラーが発生します.
  \end{itemize}
%
%  Each integer operand can be either an integer variable (with no need
%  for \cs{int_use:N}) or an integer denotation.  For example both
  整数オペランドは整数変数(\cs{int_use:N}は必要ありません)または整数
  リテラルのいずれかです.例えば
%
  \begin{verbatim}
    \int_eval:n { 5 +  4 * 3 - ( 3 + 4 * 5 ) }
  \end{verbatim}
%
%  and
  および
%
  \begin{verbatim}
    \tl_new:N  \l_my_tl
    \tl_set:Nn \l_my_tl { 5 }
    \int_new:N  \l_my_int
    \int_set:Nn \l_my_int { 4 }
    \int_eval:n { \l_my_tl + \l_my_int * 3 - ( 3 + 4 * 5 ) }
  \end{verbatim}
%
%  evaluate to $-6$ because \cs[no-index]{l_my_tl} expands to the
%  integer denotation~|5|.  As the \meta{integer expression} is fully
%  expanded from left to right during evaluation, fully expandable and
%  restricted-expandable functions can both be used, and \cs{exp_not:n}
%  and its variants have no effect while \cs{exp_not:N} may incorrectly
%  interrupt the expression.
  はいずれも$-6$と評価されます.なぜなら\cs[no-index]{l_my_tl}は整数リテ
  ラル\code{5}に展開されるためです.評価の際,整数表現\meta{integer
  expression}は左から右に向かって完全展開されるため,整数表現の中では
  完全展開可能な関数も制限付き完全展開可能な関数も使用することができます.
  また\cs{exp_not:n}およびその変種は効果がありません.これは\cs{exp_not:N}%
  が不用意に展開の邪魔をすることを防ぐためです.
%
  \begin{texnote}
%    Exactly two expansions are needed to evaluate \cs{int_eval:n}.
%    The result is \emph{not} an \meta{internal integer}, and therefore
%    requires suitable termination if used in a \TeX{}-style integer
%    assignment.
    \cs{int_eval:n}の評価にはちょうど2回の展開が必要です.評価結果は
    \meta{internal integer}では\emph{ない}ため\TeX 言語スタイルの代入を
    行う場合には適当な終了点明示が必要となります.
%
%    As all \TeX{} integers, integer operands can also be dimension or
%    skip variables, converted to integers in~\texttt{sp}, or octal
%    numbers given as \texttt{'} followed by digits other than
%    \texttt{8} and \texttt{9}, or hexadecimal numbers given as
%    |"| followed by digits or upper case letters from
%    \texttt{A} to~\texttt{F}, or the character code of some character
%    or one-character control sequence, given as \texttt{`}\meta{char}.
    \TeX における他のすべての整数と同様,整数オペランドは寸法やスキップ
    の変数(これらは\code{sp}単位の整数値に変換されます),\code{'}に
    続く8進数,|"|\eghost に続く16進数,あるいは\code{`}\meta{char}の
    形で表される文字コードでも構いません.
  \end{texnote}
\end{function}
%
\begin{function}[EXP, added = 2018-03-30]{\int_eval:w}
  \begin{syntax}
    \cs{int_eval:w} \meta{integer expression}
  \end{syntax}
%  Evaluates the \meta{integer expression} as described for
%  \cs{int_eval:n}. The end of the expression is the first token
%  encountered that cannot form part of such an expression.  If that
%  token is \cs{scan_stop:} it is removed, otherwise not.  Spaces do
%  \emph{not} terminate the expression.  However, spaces terminate
%  explict integers, and this may terminate the expression: for
%  instance, \cs{int_eval:w} \verb*|1 + 1 9| expands to \texttt{29}
%  since the digit~\texttt{9} is not part of the expression.
  整数表現\meta{integer expression}を\cs{int_eval:n}と同様に評価します.
  展開は整数表現に含まれ得ない最初のトークンを見つけた際に終了します.
  もしそのトークンが\cs{scan_stop:}だった場合は除去されますが,それ以外
  の場合はそのままになります.空白文字は展開を停止させ\emph{ない}ことに
  注意してください.ただしスペースは明示的な整数リテラルの終端を表す
  ことができるので,この性質を以て整数表現の終了点を表現することは可能
  です.例えば,\code{\cs{int_eval:w} 1 + 1 9}は\code{9}が整数表現の一部
  たり得ないことから\code{29}に展開されます.
\end{function}
%
\begin{function}[EXP, added = 2018-11-03]{\int_sign:n}
  \begin{syntax}
    \cs{int_sign:n} \Arg{intexpr}
  \end{syntax}
%  Evaluates the \meta{integer expression} then leaves $1$ or $0$ or
%  $-1$ in the input stream according to the sign of the result.
  整数表現\meta{integer expression}を評価し,その結果の符号に応じて入力
  ストリームに\code{1}, \code{0}, \code{-1}のいずれかを残します.
\end{function}
%
\begin{function}[EXP, updated = 2012-09-26]{\int_abs:n}
  \begin{syntax}
    \cs{int_abs:n} \Arg{integer expression}
  \end{syntax}
%  Evaluates the \meta{integer expression} as described for
%  \cs{int_eval:n} and leaves the absolute value of the result in
%  the input stream as an \meta{integer denotation} after two
%  expansions.
  整数表現\meta{integer expression}を\cs{int_eval:n}と同様に評価し,
  その結果の絶対値を2回展開の後に明示的な整数リテラルとして入力
  ストリームに残します.
\end{function}
%
\begin{function}[EXP, updated = 2012-09-26]{\int_div_round:nn}
  \begin{syntax}
    \cs{int_div_round:nn} \Arg{intexpr_1} \Arg{intexpr_2}
  \end{syntax}
%  Evaluates the two \meta{integer expressions} as described earlier,
%  then divides the first value by the second, and rounds the result
%  to the closest integer.  Ties are rounded away from zero.
%  Note that this is identical to using
%  |/| directly in an \meta{integer expression}. The result is left in
%  the input stream as an \meta{integer denotation} after two expansions.
  2つの整数表現をそれぞれ前述の方法で評価し,その後1つ目の値を2つ目の
  値で割ります.除算の結果は最も近い整数値に丸められます.ただし,差が
  等しい整数が2つある場合は0から遠い方に丸められます.これは整数表現の
  中で\code{/}を直接用いる場合と同じ挙動です.演算結果は2回展開の後に
  入力ストリームに明示的な整数リテラルとして残ります.
\end{function}
%
\begin{function}[EXP, updated = 2012-02-09]{\int_div_truncate:nn}
  \begin{syntax}
    \cs{int_div_truncate:nn} \Arg{intexpr_1} \Arg{intexpr_2}
  \end{syntax}
%  Evaluates the two \meta{integer expressions} as described earlier,
%  then divides the first value by the second, and rounds the result
%  towards zero.  Note that division using |/|
%  rounds to the closest integer instead.
%  The result is left in the input stream as an
%  \meta{integer denotation} after two expansions.
  2つの整数表現をそれぞれ前述の方法で評価し,その後1つ目の値を2つ目の
  値で割ります.除算の結果は0に近づく方向で最も近い整数値に丸められ
  ます.整数表現の中で\code{/}を直接用いると最も近い整数に丸められる
  ことに注意してください.演算結果は2回展開の後に入力ストリームに明示
  的な整数リテラルとして残ります.
\end{function}
%
\begin{function}[EXP, updated = 2012-09-26]{\int_max:nn, \int_min:nn}
  \begin{syntax}
    \cs{int_max:nn} \Arg{intexpr_1} \Arg{intexpr_2}
    \cs{int_min:nn} \Arg{intexpr_1} \Arg{intexpr_2}
  \end{syntax}
%  Evaluates the \meta{integer expressions} as described for
%  \cs{int_eval:n} and leaves either the larger or smaller value
%  in the input stream as an \meta{integer denotation} after two
%  expansions.
  各整数表現を\cs{int_eval:n}と同様に評価し,2回展開の後それぞれその
  結果のうち小さくない方と大きくない方を入力ストリームに残します.
\end{function}
%
\begin{function}[EXP, updated = 2012-09-26]{\int_mod:nn}
  \begin{syntax}
    \cs{int_mod:nn} \Arg{intexpr_1} \Arg{intexpr_2}
  \end{syntax}
%  Evaluates the two \meta{integer expressions} as described earlier,
%  then calculates the integer remainder of dividing the first
%  expression by the second.  This is obtained by subtracting
%  \cs{int_div_truncate:nn} \Arg{intexpr_1} \Arg{intexpr_2} times
%  \meta{intexpr_2} from \meta{intexpr_1}.  Thus, the result has the
%  same sign as \meta{intexpr_1} and its absolute value is strictly
%  less than that of \meta{intexpr_2}.  The result is left in the input
%  stream as an \meta{integer denotation} after two expansions.
  2つの整数表現をそれぞれ前述の方法で評価し,その後1つ目の値を2つ目の
  値で割った際の余りを計算します.これは\code{\cs{int_div_truncate:nn}
  \Arg{intexpr_1} \Arg{intexpr_2}}から\meta{intexpr_1}と
  \meta{intexpr_2}の積を引くことにより実現されています.したがって,
  演算結果は\meta{intexpr_1}と同じ符号をもち,その絶対値は必ず
  \meta{intexpr_2}より小さくなります.演算結果は2回展開の後に入力スト
  リームに明示的な整数リテラルとして残ります.
\end{function}
%
%\section{Creating and initialising integers}
\section{整数の作成と初期化}
%
\begin{function}{\int_new:N, \int_new:c}
  \begin{syntax}
    \cs{int_new:N} \meta{integer}
  \end{syntax}
%  Creates a new \meta{integer} or raises an error if the name is
%  already taken. The declaration is global. The \meta{integer} is
%  initially equal to $0$.
  整数\meta{integer}を作成します.与えられたトークンが既に存在する場合
  はエラーが発生します.この宣言はグローバルです.\meta{integer}は$0$に
  初期化されます.
\end{function}
%
\begin{function}[updated = 2011-10-22]{\int_const:Nn, \int_const:cn}
  \begin{syntax}
    \cs{int_const:Nn} \meta{integer} \Arg{integer expression}
  \end{syntax}
%  Creates a new constant \meta{integer} or raises an error if the name
%  is already taken. The value of the \meta{integer} is set
%  globally to the \meta{integer expression}.
  定数\meta{integer}を作成します.与えられたトークンが既に存在する場合
  はエラーが発生します.その値はグローバルに\meta{integer expression}に
  設定されます.
\end{function}
%
\begin{function}{\int_zero:N, \int_zero:c, \int_gzero:N, \int_gzero:c}
  \begin{syntax}
    \cs{int_zero:N} \meta{integer}
  \end{syntax}
%  Sets \meta{integer} to $0$.
  \meta{integer}の値を$0$に設定します.
\end{function}
%
\begin{function}[added = 2011-12-13]
  {\int_zero_new:N, \int_zero_new:c, \int_gzero_new:N, \int_gzero_new:c}
  \begin{syntax}
    \cs{int_zero_new:N} \meta{integer}
  \end{syntax}
%  Ensures that the \meta{integer} exists globally by applying
%  \cs{int_new:N} if necessary, then applies
%  \cs[index=int_zero:N]{int_(g)zero:N} to leave
%  the \meta{integer} set to zero.
  必要に応じて\cs{int_new:N}を実行することにより確実に\meta{integer}の
  存在を確実にし,その後\cs[index=int_zero:N]{int_(g)zero:N}を適用して
  \meta{integer}の値を$0$にします.
\end{function}
%
\begin{function}
  {
    \int_set_eq:NN,  \int_set_eq:cN,  \int_set_eq:Nc,  \int_set_eq:cc,
    \int_gset_eq:NN, \int_gset_eq:cN, \int_gset_eq:Nc, \int_gset_eq:cc
  }
  \begin{syntax}
    \cs{int_set_eq:NN} \meta{integer_1} \meta{integer_2}
  \end{syntax}
%  Sets the content of \meta{integer_1} equal to that of
%  \meta{integer_2}.
  \meta{integer_1}の値を\meta{integer_2}と等しい値に設定します.
\end{function}
%
\begin{function}[EXP, pTF, added=2012-03-03]
  {\int_if_exist:N, \int_if_exist:c}
  \begin{syntax}
    \cs{int_if_exist_p:N} \meta{int}
    \cs{int_if_exist:NTF} \meta{int} \Arg{true code} \Arg{false code}
  \end{syntax}
%  Tests whether the \meta{int} is currently defined.  This does not
%  check that the \meta{int} really is an integer variable.
  \meta{int}が現在定義済みかどうかをテストします.この関数は
  \meta{int}が本当に整数変数かどうかは確かめません.
\end{function}
%
%\section{Setting and incrementing integers}
\section{整数の設定とインクリメント}
%
\begin{function}[updated = 2011-10-22]
  {\int_add:Nn, \int_add:cn, \int_gadd:Nn, \int_gadd:cn}
  \begin{syntax}
    \cs{int_add:Nn} \meta{integer} \Arg{integer expression}
  \end{syntax}
%  Adds the result of the \meta{integer expression} to the current
%  content of the \meta{integer}.
  \meta{integer}の現在の値に\meta{integer expression}の評価結果を
  加えます.
\end{function}
%
\begin{function}{\int_decr:N, \int_decr:c, \int_gdecr:N, \int_gdecr:c}
  \begin{syntax}
    \cs{int_decr:N} \meta{integer}
  \end{syntax}
%  Decreases the value stored in \meta{integer} by $1$.
  \meta{integer}の値を$1$減じます.
\end{function}
%
\begin{function}{\int_incr:N, \int_incr:c, \int_gincr:N, \int_gincr:c}
  \begin{syntax}
    \cs{int_incr:N} \meta{integer}
  \end{syntax}
%  Increases the value stored in \meta{integer} by $1$.
  \meta{integer}の値を$1$増やします.
\end{function}
%
\begin{function}[updated = 2011-10-22]
  {\int_set:Nn, \int_set:cn, \int_gset:Nn, \int_gset:cn}
  \begin{syntax}
    \cs{int_set:Nn} \meta{integer} \Arg{integer expression}
  \end{syntax}
%  Sets \meta{integer} to the value of \meta{integer expression},
%  which must evaluate to an integer (as described for
%  \cs{int_eval:n}).
  \meta{integer}の値を\meta{integer expression}に設定します.
  \meta{integer expression}の評価結果は必ず整数になる必要があります
  (\cs{int_eval:n}と同様です).
\end{function}
%
\begin{function}[updated = 2011-10-22]
  {\int_sub:Nn, \int_sub:cn, \int_gsub:Nn, \int_gsub:cn}
  \begin{syntax}
    \cs{int_sub:Nn} \meta{integer} \Arg{integer expression}
  \end{syntax}
%  Subtracts the result of the \meta{integer expression} from the
%  current content of the \meta{integer}.
  \meta{integer}の現在の値から\meta{integer expression}の評価結果を
  減じます.
\end{function}
%
%\section{Using integers}
\section{整数の使用}
%
\begin{function}[updated = 2011-10-22, EXP]{\int_use:N, \int_use:c}
  \begin{syntax}
    \cs{int_use:N} \meta{integer}
  \end{syntax}
%  Recovers the content of an \meta{integer} and places it directly
%  in the input stream. An error is raised if the variable does
%  not exist or if it is invalid. Can be omitted in places where an
%  \meta{integer} is required (such as in the first and third arguments
%  of \cs{int_compare:nNnTF}).
  \meta{integer}の値を取り出して直接入力ストリームに置きます.変数
  \meta{integer}が存在しない場合や不正な場合はエラーが発生します.
  \meta{integer}が要求されている場所では省略できます(例えば
  \cs{int_compare:nNnTF}関数の第1, 3引数).
  \begin{texnote}
%    \cs{int_use:N} is the \TeX{} primitive \tn{the}: this is one of
%    several \LaTeX3 names for this primitive.
    \cs{int_use:N}は\TeX プリミティブの\tn{the}に相当します.\tn{the}%
    の\LaTeX3における名称が複数ありますが,そのうちの1つです.
  \end{texnote}
\end{function}
%
\section{Integer expression conditionals}
%
\begin{function}[EXP,pTF]{\int_compare:nNn}
  \begin{syntax}
    \cs{int_compare_p:nNn} \Arg{intexpr_1} \meta{relation} \Arg{intexpr_2} \\
    \cs{int_compare:nNnTF}
    ~~\Arg{intexpr_1} \meta{relation} \Arg{intexpr_2}
    ~~\Arg{true code} \Arg{false code}
  \end{syntax}
  This function first evaluates each of the \meta{integer expressions}
  as described for \cs{int_eval:n}. The two results are then
  compared using the \meta{relation}:
  \begin{center}
    \begin{tabular}{ll}
      Equal                 & |=| \\
      Greater than          & |>| \\
      Less than             & |<| \\
    \end{tabular}
  \end{center}
  This function is less flexible than \cs{int_compare:nTF} but around
  $5$~times faster.
\end{function}
%
\begin{function}[updated = 2013-01-13, EXP, pTF]{\int_compare:n}
  \begin{syntax}
    \cs{int_compare_p:n} \\
    ~~\{ \\
    ~~~~\meta{intexpr_1} \meta{relation_1} \\
    ~~~~\ldots{} \\
    ~~~~\meta{intexpr_N} \meta{relation_N} \\
    ~~~~\meta{intexpr_{N+1}} \\
    ~~\} \\
    \cs{int_compare:nTF}
    ~~\{ \\
    ~~~~\meta{intexpr_1} \meta{relation_1} \\
    ~~~~\ldots{} \\
    ~~~~\meta{intexpr_N} \meta{relation_N} \\
    ~~~~\meta{intexpr_{N+1}} \\
    ~~\} \\
    ~~\Arg{true code} \Arg{false code}
  \end{syntax}
  This function evaluates the \meta{integer expressions} as described
  for \cs{int_eval:n} and compares consecutive result using the
  corresponding \meta{relation}, namely it compares \meta{intexpr_1}
  and \meta{intexpr_2} using the \meta{relation_1}, then
  \meta{intexpr_2} and \meta{intexpr_3} using the \meta{relation_2},
  until finally comparing \meta{intexpr_N} and \meta{intexpr_{N+1}}
  using the \meta{relation_N}.  The test yields \texttt{true} if all
  comparisons are \texttt{true}.  Each \meta{integer expression} is
  evaluated only once, and the evaluation is lazy, in the sense that
  if one comparison is \texttt{false}, then no other \meta{integer
    expression} is evaluated and no other comparison is performed.
  The \meta{relations} can be any of the following:
  \begin{center}
    \begin{tabular}{ll}
      Equal                    & |=| or |==| \\
      Greater than or equal to & |>=|        \\
      Greater than             & |>|         \\
      Less than or equal to    & |<=|        \\
      Less than                & |<|         \\
      Not equal                & |!=|        \\
    \end{tabular}
  \end{center}
  This function is more flexible than \cs{int_compare:nNnTF} but
  around $5$~times slower.
\end{function}
%
\begin{function}[added = 2013-07-24, EXP, noTF]{\int_case:nn}
  \begin{syntax}
    \cs{int_case:nnTF} \Arg{test integer expression} \\
    ~~|{| \\
    ~~~~\Arg{intexpr case_1} \Arg{code case_1} \\
    ~~~~\Arg{intexpr case_2} \Arg{code case_2} \\
    ~~~~\ldots \\
    ~~~~\Arg{intexpr case_n} \Arg{code case_n} \\
    ~~|}| \\
    ~~\Arg{true code}
    ~~\Arg{false code}
  \end{syntax}
  This function evaluates the \meta{test integer expression} and
  compares this in turn to each of the
  \meta{integer expression cases}. If the two are equal then the
  associated \meta{code} is left in the input stream
  and other cases are discarded. If any of the
  cases are matched, the \meta{true code} is also inserted into the
  input stream (after the code for the appropriate case), while if none
  match then the \meta{false code} is inserted. The function
  \cs{int_case:nn}, which does nothing if there is no match, is also
  available. For example
  \begin{verbatim}
    \int_case:nnF
      { 2 * 5 }
      {
        { 5 }       { Small }
        { 4 + 6 }   { Medium }
        { -2 * 10 } { Negative }
      }
      { No idea! }
  \end{verbatim}
  leaves \enquote{\texttt{Medium}} in the input stream.
\end{function}
%
\begin{function}[EXP,pTF]{\int_if_even:n, \int_if_odd:n}
  \begin{syntax}
    \cs{int_if_odd_p:n} \Arg{integer expression}
    \cs{int_if_odd:nTF} \Arg{integer expression}
    ~~\Arg{true code} \Arg{false code}
  \end{syntax}
  This function first evaluates the \meta{integer expression}
  as described for \cs{int_eval:n}. It then evaluates if this
  is odd or even, as appropriate.
\end{function}
%
\section{Integer expression loops}
%
\begin{function}[rEXP]{\int_do_until:nNnn}
  \begin{syntax}
     \cs{int_do_until:nNnn} \Arg{intexpr_1} \meta{relation} \Arg{intexpr_2} \Arg{code}
  \end{syntax}
  Places the \meta{code} in the input stream for \TeX{} to process, and
  then evaluates the relationship between the two
  \meta{integer expressions} as described for \cs{int_compare:nNnTF}.
  If the test is \texttt{false} then the \meta{code} is inserted
  into the input stream again and a loop occurs until the
  \meta{relation} is \texttt{true}.
\end{function}
%
\begin{function}[rEXP]{\int_do_while:nNnn}
  \begin{syntax}
     \cs{int_do_while:nNnn} \Arg{intexpr_1} \meta{relation} \Arg{intexpr_2} \Arg{code}
  \end{syntax}
  Places the \meta{code} in the input stream for \TeX{} to process, and
  then evaluates the relationship between the two
  \meta{integer expressions} as described for \cs{int_compare:nNnTF}.
  If the test is \texttt{true} then the \meta{code} is inserted
  into the input stream again and a loop occurs until the
  \meta{relation} is \texttt{false}.
\end{function}
%
\begin{function}[rEXP]{\int_until_do:nNnn}
  \begin{syntax}
     \cs{int_until_do:nNnn} \Arg{intexpr_1} \meta{relation} \Arg{intexpr_2} \Arg{code}
  \end{syntax}
  Evaluates the relationship between the two \meta{integer expressions}
  as described for \cs{int_compare:nNnTF}, and then places the
  \meta{code} in the input stream if the \meta{relation} is
  \texttt{false}. After the \meta{code} has been processed by \TeX{} the
  test is repeated, and a loop occurs until the test is
  \texttt{true}.
\end{function}
%
\begin{function}[rEXP]{\int_while_do:nNnn}
  \begin{syntax}
     \cs{int_while_do:nNnn} \Arg{intexpr_1} \meta{relation} \Arg{intexpr_2} \Arg{code}
  \end{syntax}
  Evaluates the relationship between the two \meta{integer expressions}
  as described for \cs{int_compare:nNnTF}, and then places the
  \meta{code} in the input stream if the \meta{relation} is
  \texttt{true}. After the \meta{code} has been processed by \TeX{} the
  test is repeated, and a loop occurs until the test is
  \texttt{false}.
\end{function}
%
\begin{function}[updated = 2013-01-13, rEXP]{\int_do_until:nn}
  \begin{syntax}
     \cs{int_do_until:nn} \Arg{integer relation} \Arg{code}
  \end{syntax}
  Places the \meta{code} in the input stream for \TeX{} to process, and
  then evaluates the \meta{integer relation}
  as described for \cs{int_compare:nTF}.
  If the test is \texttt{false} then the \meta{code} is inserted
  into the input stream again and a loop occurs until the
  \meta{relation} is \texttt{true}.
\end{function}
%
\begin{function}[updated = 2013-01-13, rEXP]{\int_do_while:nn}
  \begin{syntax}
     \cs{int_do_while:nn} \Arg{integer relation} \Arg{code}
  \end{syntax}
  Places the \meta{code} in the input stream for \TeX{} to process, and
  then evaluates the \meta{integer relation}
  as described for \cs{int_compare:nTF}.
  If the test is \texttt{true} then the \meta{code} is inserted
  into the input stream again and a loop occurs until the
  \meta{relation} is \texttt{false}.
\end{function}
%
\begin{function}[updated = 2013-01-13, rEXP]{\int_until_do:nn}
  \begin{syntax}
     \cs{int_until_do:nn} \Arg{integer relation} \Arg{code}
  \end{syntax}
  Evaluates the \meta{integer relation}
  as described for \cs{int_compare:nTF}, and then places the
  \meta{code} in the input stream if the \meta{relation} is
  \texttt{false}. After the \meta{code} has been processed by \TeX{} the
  test is repeated, and a loop occurs until the test is
  \texttt{true}.
\end{function}
%
\begin{function}[updated = 2013-01-13, rEXP]{\int_while_do:nn}
  \begin{syntax}
     \cs{int_while_do:nn} \Arg{integer relation} \Arg{code}
  \end{syntax}
  Evaluates the \meta{integer relation}
  as described for \cs{int_compare:nTF}, and then places the
  \meta{code} in the input stream if the \meta{relation} is
  \texttt{true}. After the \meta{code} has been processed by \TeX{} the
  test is repeated, and a loop occurs until the test is
  \texttt{false}.
\end{function}
%
\section{Integer step functions}
%
\begin{function}[added = 2012-06-04, updated = 2018-04-22, rEXP]
  {\int_step_function:nN, \int_step_function:nnN, \int_step_function:nnnN}
  \begin{syntax}
    \cs{int_step_function:nN} \Arg{final value} \meta{function}
    \cs{int_step_function:nnN} \Arg{initial value} \Arg{final value} \meta{function}
    \cs{int_step_function:nnnN} \Arg{initial value} \Arg{step} \Arg{final value} \meta{function}
  \end{syntax}
  This function first evaluates the \meta{initial value}, \meta{step}
  and \meta{final value}, all of which should be integer expressions.
  The \meta{function} is then placed in front of each \meta{value}
  from the \meta{initial value} to the \meta{final value} in turn
  (using \meta{step} between each \meta{value}).  The \meta{step} must
  be non-zero.  If the \meta{step} is positive, the loop stops when
  the \meta{value} becomes larger than the \meta{final value}.  If the
  \meta{step} is negative, the loop stops when the \meta{value}
  becomes smaller than the \meta{final value}.  The \meta{function}
  should absorb one numerical argument. For example
  \begin{verbatim}
    \cs_set:Npn \my_func:n #1 { [I~saw~#1] \quad }
    \int_step_function:nnnN { 1 } { 1 } { 5 } \my_func:n
  \end{verbatim}
  would print
  \begin{quote}
    [I saw 1] \quad
    [I saw 2] \quad
    [I saw 3] \quad
    [I saw 4] \quad
    [I saw 5] \quad
  \end{quote}
%
  The functions \cs{int_step_function:nN} and \cs{int_step_function:nnN}
  both use a fixed \meta{step} of $1$, and in the case of
  \cs{int_step_function:nN} the \meta{initial value} is also fixed as
  $1$. These functions are provided as simple short-cuts for code clarity.
\end{function}
%
\begin{function}[added = 2012-06-04, updated = 2018-04-22]
  {\int_step_inline:nn, \int_step_inline:nnn, \int_step_inline:nnnn}
  \begin{syntax}
    \cs{int_step_inline:nn} \Arg{final value} \Arg{code}
    \cs{int_step_inline:nnn} \Arg{initial value} \Arg{final value} \Arg{code}
    \cs{int_step_inline:nnnn} \Arg{initial value} \Arg{step} \Arg{final value} \Arg{code}
  \end{syntax}
  This function first evaluates the \meta{initial value}, \meta{step}
  and \meta{final value}, all of which should be integer expressions.
  Then for each \meta{value} from the \meta{initial value} to the
  \meta{final value} in turn (using \meta{step} between each
  \meta{value}), the \meta{code} is inserted into the input stream
  with |#1| replaced by the current \meta{value}.  Thus the
  \meta{code} should define a function of one argument~(|#1|).
%
  The functions \cs{int_step_inline:nn} and \cs{int_step_inline:nnn}
  both use a fixed \meta{step} of $1$, and in the case of
  \cs{int_step_inline:nn} the \meta{initial value} is also fixed as
  $1$. These functions are provided as simple short-cuts for code clarity.
\end{function}
%
\begin{function}[added = 2012-06-04, updated = 2018-04-22]
  {\int_step_variable:nNn, \int_step_variable:nnNn, \int_step_variable:nnnNn}
  \begin{syntax}
    \cs{int_step_variable:nNn} \Arg{final value} \meta{tl~var} \Arg{code}
    \cs{int_step_variable:nnNn} \Arg{initial value} \Arg{final value} \meta{tl~var} \Arg{code}
    \cs{int_step_variable:nnnNn} \Arg{initial value} \Arg{step} \Arg{final value} \meta{tl~var} \Arg{code}
  \end{syntax}
  This function first evaluates the \meta{initial value}, \meta{step}
  and \meta{final value}, all of which should be integer expressions.
  Then for each \meta{value} from the \meta{initial value} to the
  \meta{final value} in turn (using \meta{step} between each
  \meta{value}), the \meta{code} is inserted into the input stream,
  with the \meta{tl~var} defined as the current \meta{value}.  Thus
  the \meta{code} should make use of the \meta{tl~var}.
%
  The functions \cs{int_step_variable:nNn} and \cs{int_step_variable:nnNn}
  both use a fixed \meta{step} of $1$, and in the case of
  \cs{int_step_variable:nNn} the \meta{initial value} is also fixed as
  $1$. These functions are provided as simple short-cuts for code clarity.
\end{function}
%
\section{Formatting integers}
%
Integers can be placed into the output stream with formatting. These
conversions apply to any integer expressions.
%
\begin{function}[updated = 2011-10-22, EXP]{\int_to_arabic:n}
  \begin{syntax}
    \cs{int_to_arabic:n} \Arg{integer expression}
  \end{syntax}
  Places the value of the \meta{integer expression} in the input
  stream as digits, with category code $12$ (other).
\end{function}
%
\begin{function}[updated = 2011-09-17, EXP]{\int_to_alph:n, \int_to_Alph:n}
  \begin{syntax}
    \cs{int_to_alph:n} \Arg{integer expression}
  \end{syntax}
  Evaluates the \meta{integer expression} and converts the result
  into a series of letters, which are then left in the input stream.
  The conversion rule uses the $26$ letters of the English
  alphabet, in order, adding letters when necessary to increase the total
  possible range of representable numbers. Thus
  \begin{verbatim}
    \int_to_alph:n { 1 }
  \end{verbatim}
  places |a| in the input stream,
  \begin{verbatim}
    \int_to_alph:n { 26 }
  \end{verbatim}
  is represented as |z| and
  \begin{verbatim}
    \int_to_alph:n { 27 }
  \end{verbatim}
  is converted to |aa|. For conversions using other alphabets, use
  \cs{int_to_symbols:nnn} to define an alphabet-specific
  function. The basic \cs{int_to_alph:n} and \cs{int_to_Alph:n}
  functions should not be modified.
  The resulting tokens are digits with category code $12$ (other) and
  letters with category code $11$ (letter).
\end{function}
%
\begin{function}[updated = 2011-09-17, EXP]{\int_to_symbols:nnn}
  \begin{syntax}
    \cs{int_to_symbols:nnn}
    ~~\Arg{integer expression} \Arg{total symbols}
    ~~\Arg{value to symbol mapping}
  \end{syntax}
  This is the low-level function for conversion of an
  \meta{integer expression} into a symbolic form (often
  letters). The \meta{total symbols} available should be given
  as an integer expression. Values are actually converted to symbols
  according to the \meta{value to symbol mapping}. This should be given
  as \meta{total symbols} pairs of entries, a number and the
  appropriate symbol. Thus the \cs{int_to_alph:n} function is defined
  as
  \begin{verbatim}
    \cs_new:Npn \int_to_alph:n #1
      {
        \int_to_symbols:nnn {#1} { 26 }
          {
            {  1 } { a }
            {  2 } { b }
            ...
            { 26 } { z }
          }
      }
  \end{verbatim}
\end{function}
%
\begin{function}[added = 2014-02-11, EXP]{\int_to_bin:n}
  \begin{syntax}
    \cs{int_to_bin:n} \Arg{integer expression}
  \end{syntax}
  Calculates the value of the \meta{integer expression} and places
  the binary representation of the result in the input stream.
\end{function}
%
\begin{function}[added = 2014-02-11, EXP]{\int_to_hex:n, \int_to_Hex:n}
  \begin{syntax}
    \cs{int_to_hex:n} \Arg{integer expression}
  \end{syntax}
  Calculates the value of the \meta{integer expression} and places
  the hexadecimal (base~$16$) representation of the result in the
  input stream. Letters are used for digits beyond~$9$: lower
  case letters for \cs{int_to_hex:n} and upper case ones for
  \cs{int_to_Hex:n}.
  The resulting tokens are digits with category code $12$ (other) and
  letters with category code $11$ (letter).
\end{function}
%
\begin{function}[added = 2014-02-11, EXP]{\int_to_oct:n}
  \begin{syntax}
    \cs{int_to_oct:n} \Arg{integer expression}
  \end{syntax}
  Calculates the value of the \meta{integer expression} and places
  the octal (base~$8$) representation of the result in the input
  stream.
  The resulting tokens are digits with category code $12$ (other) and
  letters with category code $11$ (letter).
\end{function}
%
\begin{function}[updated = 2014-02-11, EXP]
  {\int_to_base:nn, \int_to_Base:nn}
  \begin{syntax}
    \cs{int_to_base:nn} \Arg{integer expression} \Arg{base}
  \end{syntax}
  Calculates the value of the \meta{integer expression} and
  converts it into the appropriate representation in the \meta{base};
  the later may be given as an integer expression. For bases greater
  than $10$ the higher \enquote{digits} are represented by
  letters from the English alphabet:  lower
  case letters for \cs{int_to_base:n} and upper case ones for
  \cs{int_to_Base:n}.
  The maximum \meta{base} value is $36$.
  The resulting tokens are digits with category code $12$ (other) and
  letters with category code $11$ (letter).
  \begin{texnote}
    This is a generic version of \cs{int_to_bin:n}, \emph{etc.}
  \end{texnote}
\end{function}
%
\begin{function}[updated = 2011-10-22, rEXP]{\int_to_roman:n, \int_to_Roman:n}
  \begin{syntax}
    \cs{int_to_roman:n} \Arg{integer expression}
  \end{syntax}
  Places the value of the \meta{integer expression} in the input
  stream as Roman numerals, either lower case (\cs{int_to_roman:n}) or
  upper case (\cs{int_to_Roman:n}).  If the value is negative or zero,
  the output is empty.  The Roman numerals are letters with category
  code $11$ (letter).  The letters used are |mdclxvi|, repeated as
  needed: the notation with bars (such as $\bar{\mbox{v}}$ for $5000$)
  is \emph{not} used.  For instance \cs{int_to_roman:n} |{| 8249 |}|
  expands to |mmmmmmmmccxlix|.
\end{function}
%
\section{Converting from other formats to integers}
%
\begin{function}[updated = 2014-08-25, EXP]{\int_from_alph:n}
  \begin{syntax}
    \cs{int_from_alph:n} \Arg{letters}
  \end{syntax}
  Converts the \meta{letters} into the integer (base~$10$)
  representation and leaves this in the input stream.  The
  \meta{letters} are first converted to a string, with no expansion.
  Lower and upper case letters from the English alphabet may be used,
  with \enquote{a} equal to $1$ through to \enquote{z} equal to $26$.
  The function also accepts a leading sign, made of |+| and~|-|.  This
  is the inverse function of \cs{int_to_alph:n} and
  \cs{int_to_Alph:n}.
\end{function}
%
\begin{function}[added = 2014-02-11, updated = 2014-08-25, EXP]
  {\int_from_bin:n}
  \begin{syntax}
    \cs{int_from_bin:n} \Arg{binary number}
  \end{syntax}
  Converts the \meta{binary number} into the integer (base~$10$)
  representation and leaves this in the input stream.
  The \meta{binary number} is first converted to a string, with no
  expansion.  The function accepts a leading sign, made of |+|
  and~|-|, followed by binary digits.  This is the inverse function
  of \cs{int_to_bin:n}.
\end{function}
%
\begin{function}[added = 2014-02-11, updated = 2014-08-25, EXP]
  {\int_from_hex:n}
  \begin{syntax}
    \cs{int_from_hex:n} \Arg{hexadecimal number}
  \end{syntax}
  Converts the \meta{hexadecimal number} into the integer (base~$10$)
  representation and leaves this in the input stream.  Digits greater
  than $9$ may be represented in the \meta{hexadecimal number} by
  upper or lower case letters.  The \meta{hexadecimal number} is first
  converted to a string, with no expansion.  The function also accepts
  a leading sign, made of |+| and~|-|.  This is the inverse function
  of \cs{int_to_hex:n} and \cs{int_to_Hex:n}.
\end{function}
%
\begin{function}[added = 2014-02-11, updated = 2014-08-25, EXP]
  {\int_from_oct:n}
  \begin{syntax}
    \cs{int_from_oct:n} \Arg{octal number}
  \end{syntax}
  Converts the \meta{octal number} into the integer (base~$10$)
  representation and leaves this in the input stream.
  The \meta{octal number} is first converted to a string, with no
  expansion.  The function accepts a leading sign, made of |+|
  and~|-|, followed by octal digits.  This is the inverse function
  of \cs{int_to_oct:n}.
\end{function}
%
\begin{function}[updated = 2014-08-25, updated = 2014-08-25, EXP]
  {\int_from_roman:n}
  \begin{syntax}
    \cs{int_from_roman:n} \Arg{roman numeral}
  \end{syntax}
  Converts the \meta{roman numeral} into the integer (base~$10$)
  representation and leaves this in the input stream.  The \meta{roman
    numeral} is first converted to a string, with no expansion.  The
  \meta{roman numeral} may be in upper or lower case; if the numeral
  contains characters besides |mdclxvi| or |MDCLXVI| then the
  resulting value is $-1$.  This is the inverse function of
  \cs{int_to_roman:n} and \cs{int_to_Roman:n}.
\end{function}
%
\begin{function}[updated = 2014-08-25, EXP]{\int_from_base:nn}
  \begin{syntax}
    \cs{int_from_base:nn} \Arg{number} \Arg{base}
  \end{syntax}
  Converts the \meta{number} expressed in \meta{base} into the
  appropriate value in base $10$.  The \meta{number} is first
  converted to a string, with no expansion.  The \meta{number} should
  consist of digits and letters (either lower or upper case), plus
  optionally a leading sign. The maximum \meta{base} value is $36$.
  This is the inverse function of \cs{int_to_base:nn} and
  \cs{int_to_Base:nn}.
\end{function}
%
\section{Random integers}
%
\begin{function}[EXP, added = 2016-12-06, updated = 2018-04-27]{\int_rand:nn}
  \begin{syntax}
    \cs{int_rand:nn} \Arg{intexpr_1} \Arg{intexpr_2}
  \end{syntax}
  Evaluates the two \meta{integer expressions} and produces a
  pseudo-random number between the two (with bounds included).
  This is not available in older versions of \XeTeX{}.
\end{function}
%
\begin{function}[EXP, added = 2018-05-05]{\int_rand:n}
  \begin{syntax}
    \cs{int_rand:n} \Arg{intexpr}
  \end{syntax}
  Evaluates the \meta{integer expression} then produces a
  pseudo-random number between $1$ and the \meta{intexpr} (included).
  This is not available in older versions of \XeTeX{}.
\end{function}
%
\section{Viewing integers}
%
\begin{function}{\int_show:N, \int_show:c}
  \begin{syntax}
    \cs{int_show:N} \meta{integer}
  \end{syntax}
  Displays the value of the \meta{integer} on the terminal.
\end{function}
%
\begin{function}[added = 2011-11-22, updated = 2015-08-07]{\int_show:n}
  \begin{syntax}
    \cs{int_show:n} \Arg{integer expression}
  \end{syntax}
  Displays the result of evaluating the \meta{integer expression}
  on the terminal.
\end{function}
%
\begin{function}[added = 2014-08-22, updated = 2015-08-03]{\int_log:N, \int_log:c}
  \begin{syntax}
    \cs{int_log:N} \meta{integer}
  \end{syntax}
  Writes the value of the \meta{integer} in the log file.
\end{function}
%
\begin{function}[added = 2014-08-22, updated = 2015-08-07]{\int_log:n}
  \begin{syntax}
    \cs{int_log:n} \Arg{integer expression}
  \end{syntax}
  Writes the result of evaluating the \meta{integer expression}
  in the log file.
\end{function}
%
\section{Constant integers}
%
\begin{variable}[added = 2018-05-07]{\c_zero_int, \c_one_int}
  Integer values used with primitive tests and assignments: their
  self-terminating nature makes these more convenient and faster than
  literal numbers.
\end{variable}
%
\begin{variable}{\c_max_int}
  The maximum value that can be stored as an integer.
\end{variable}
%
\begin{variable}{\c_max_register_int}
  Maximum number of registers.
\end{variable}
%
\begin{variable}{\c_max_char_int}
  Maximum character code completely supported by the engine.
\end{variable}
%
\section{Scratch integers}
%
\begin{variable}{\l_tmpa_int, \l_tmpb_int}
  Scratch integer for local assignment. These are never used by
  the kernel code, and so are safe for use with any \LaTeX3-defined
  function. However, they may be overwritten by other non-kernel
  code and so should only be used for short-term storage.
\end{variable}
%
\begin{variable}{\g_tmpa_int, \g_tmpb_int}
  Scratch integer for global assignment. These are never used by
  the kernel code, and so are safe for use with any \LaTeX3-defined
  function. However, they may be overwritten by other non-kernel
  code and so should only be used for short-term storage.
\end{variable}
%
\subsection{Direct number expansion}
%
\begin{function}[EXP, added = 2018-03-27]{\int_value:w}
  \begin{syntax}
    \cs{int_value:w} \meta{integer}
    \cs{int_value:w} \meta{integer denotation} \meta{optional space}
  \end{syntax}
  Expands the following tokens until an \meta{integer} is formed, and
  leaves a normalized form (no leading sign except for negative
  numbers, no leading digit~|0| except for zero) in the input stream
  as category code $12$ (other) characters.  The \meta{integer} can
  consist of any number of signs (with intervening spaces) followed
  by
  \begin{itemize}
    \item an integer variable (in fact, any \TeX{} register except
      \tn{toks}) or
    \item explicit digits (or by |'|\meta{octal digits} or |"|\meta{hexadecimal digits} or |`|\meta{character}).
  \end{itemize}
  In this last case expansion stops once a non-digit is found; if that is a
  space it is removed as in \texttt{f}-expansion, and so \cs{exp_stop_f:}
  may be employed as an end marker. Note that protected functions
  \emph{are} expanded by this process.
%
  This function requires exactly one expansion to produce a value, and so
  is suitable for use in cases where a number is required \enquote{directly}.
  In general, \cs{int_eval:n} is the preferred approach to generating
  numbers.
  \begin{texnote}
    This is the \TeX{} primitive \tn{number}.
  \end{texnote}
\end{function}
%
\section{Primitive conditionals}
%
\begin{function}[EXP]{\if_int_compare:w}
  \begin{syntax}
    \cs{if_int_compare:w} \meta{integer_1} \meta{relation} \meta{integer_2}
    ~~\meta{true code}
    \cs{else:}
    ~~\meta{false code}
    \cs{fi:}
  \end{syntax}
  Compare two integers using \meta{relation}, which must be one of
  |=|, |<| or |>| with category code $12$.
  The \cs{else:} branch is optional.
  \begin{texnote}
    These are both names for the \TeX{} primitive \tn{ifnum}.
  \end{texnote}
\end{function}
%
\begin{function}[EXP]{\if_case:w, \or:}
  \begin{syntax}
    \cs{if_case:w} \meta{integer} \meta{case_0}
    ~~\cs{or:} \meta{case_1}
    ~~\cs{or:} |...|
    ~~\cs{else:} \meta{default}
    \cs{fi:}
  \end{syntax}
  Selects a case to execute based on the value of the \meta{integer}. The
  first case (\meta{case_0}) is executed if \meta{integer} is $0$, the second
  (\meta{case_1}) if the \meta{integer} is $1$, \emph{etc.} The
  \meta{integer} may be a literal, a constant or an integer
  expression (\emph{e.g.}~using \cs{int_eval:n}).
  \begin{texnote}
    These are the \TeX{} primitives \tn{ifcase} and \tn{or}.
  \end{texnote}
\end{function}
%
\begin{function}[EXP]{\if_int_odd:w}
  \begin{syntax}
    \cs{if_int_odd:w} \meta{tokens}  \meta{optional space}
    ~~\meta{true code}
    \cs{else:}
    ~~\meta{true code}
    \cs{fi:}
  \end{syntax}
  Expands \meta{tokens} until a non-numeric token or a space is found, and
  tests whether the resulting \meta{integer} is odd. If so, \meta{true code}
  is executed. The \cs{else:} branch is optional.
  \begin{texnote}
    This is the \TeX{} primitive \tn{ifodd}.
  \end{texnote}
\end{function}
%
\end{documentation}
%
\PrintIndex
%
\end{document}

%\DocInput{l3flag.dtx}
%\DocInput{l3prg.dtx}
%\DocInput{l3sys.dtx}
%\DocInput{l3clist.dtx}
%\DocInput{l3token.dtx}
%\DocInput{l3prop.dtx}
%\DocInput{l3msg.dtx}
%\DocInput{l3file.dtx}
%\DocInput{l3skip.dtx}
%\DocInput{l3keys.dtx}
%\DocInput{l3intarray.dtx}
%\DocInput{l3fp.dtx}
%\DocInput{l3fparray.dtx}
%\DocInput{l3sort.dtx}
%\DocInput{l3tl-analysis.dtx}
%\DocInput{l3regex.dtx}
%\DocInput{l3box.dtx}
%\DocInput{l3coffins.dtx}
%\DocInput{l3color-base.dtx}
%\DocInput{l3luatex.dtx}
%\DocInput{l3unicode.dtx}
%\DocInput{l3text.dtx}
%\DocInput{l3legacy.dtx}
%\DocInput{l3candidates.dtx}

% 終了部
\clearpage

\begingroup
  \def\endash{--}
  \catcode`\-\active
  \def-{\futurelet\temp\indexdash}
  \def\indexdash{\ifx\temp-\endash\fi}
  \DelayPrintIndex
\endgroup

\end{document}
